\usepackage[T1]{fontenc}
\usepackage[french]{babel}
%\usepackage[utf8x]{inputenc}
\usepackage[utf8]{inputenc}
\usepackage[left=3cm,right=3cm,top=2cm,bottom=3cm]{geometry}
\usepackage{enumitem}
\frenchbsetup{StandardLists=true}
\usepackage{supertabular}
\newframedtheorem
	

\usepackage[colored]{shadethm}
\definecolor{shadethmcolor}{rgb}{0.8,0.8,0.8}% couleur du fond
\definecolor{shaderulecolor}{rgb}{0,0,1}% couleur de l'encadré

%\usepackage[english,french]{babel}
%\usepackage[latin1]{inputenc}
%\usepackage{natbib}
\usepackage{amssymb}
\usepackage{multicol}
\usepackage[fleqn]{amsmath}
\usepackage{epsfig}
\usepackage[normalem]{ulem}
\usepackage{verbatim}
\usepackage{graphicx}
\usepackage{graphics}
\usepackage{url} % pour insérer des url
\usepackage{color}
\usepackage{xcolor}
\usepackage{bbm}
\usepackage{bm}
\usepackage{dsfont}
\usepackage{amsmath,amsfonts,times,latexsym,comment,times}
\usepackage{color,epsfig,rotating}
\newcommand{\ds}{\displaystyle}
\newcommand{\bce}{\begin{center}}
\newcommand{\ece}{\end{center}}
%\usepackage{mprocl}

\setitemize[0]{font=\bfseries, label=\textbullet}  

\def\bX{\mathbf{X}} % raccourcis pour un vecteur aléatoire
\def\bY{\mathbf{Y}}
\def\bZ{\mathbf{Z}}
\def\vec{\mathbf}
\def\cN{{\cal{N}}}
\def\cH{{\cal{H}}}
\def\cG{{\cal{G}}}
\def\Expect{\mathbbm{E}}
\def\btheta{{\boldsymbol{\theta}}}  % raccourci pour le vecteur de paramètres usuel 
\def\bbeta{{\boldsymbol{\beta}}}  % raccourci pour le vecteur de paramètres usuel 

\def\bpsi{{\boldsymbol{\psi}}}  % raccourci pour le vecteur de paramètres usuel 
 

\def\ba{\mathbf{a}}  % raccourcis pour les suites de vecteurs impliqués dans la
\def\bb{\mathbf{b}}  % démonstration des théorèmes limites

\def\barchi{\bar\chi}  % raccourci pour un coefficient de dépendance extrême

\def\brho{{\boldsymbol{\rho}}}  % raccourci pour une matrice de corrélation

\def\sign{\text{sign}} % raccourci pour le signe d'un scalaire
\def\supp{\text{Supp}} % raccourci pour le support d'une densité
\def\trace{\text{tr}} % raccourci pour la trace d'une matrice 
\def\diag{\text{diag}} % raccourci pour le vecteur diagonal d'une matrice 

\def\grad{\nabla} 
\def\bx{\mathbf{x}}
\def\by{\mathbf{y}}
\def\bz{\mathbf{z}}
\def\bp{\mathbf{p}}
\newcommand{\MRTF}{\mbox{MRTF}}
\newcommand{\mttf}{\mbox{mttf}}
\newcommand{\mode}{\mbox{md}}
\newcommand{\sS}{\mbox{S}}
\newcommand{\LL}{\ell}
\newcommand{\DAC}{\mbox{DAC}}
\newcommand{\CV}{\text{CV}}  % un raccourci pour le coefficient de variation
\newcommand{\D}{\mbox{D}}
\newcommand{\R}{I\!\!R}
\newcommand{\N}{I\!\!N}
\newcommand{\Q}{\mathbbm{Q}}
\newcommand{\I}{\mathds{1}}
\newcommand{\C}{C}
\newcommand{\Pp}{\mathbbm{P}}
\newcommand{\E}{\mathbbm{E}}
\newcommand{\V}{\mathbbm{V}} 
\newcommand{\Var}{\mathbbm{V}} 
\newcommand{\Cov}{\mathbbm{C}{ov}} 
\newcommand{\1}{\mathbbm{1}}
\newcommand{\Med}{\mbox{Med}}
\newcommand{\Mod}{\mbox{Mod}}
\newcommand{\Md}{\mbox{M}_d}
\newcommand{\Card}{\mbox{Card}}
\newcommand{\DIP}{\mbox{Dip}}
\newcommand{\Supp}{\mbox{Supp}}
\newcommand{\Pm}{\mathbbm{P}}

\def\sign{\text{sign}} % raccourci pour le signe d'un scalaire
\def\supp{\text{Supp}} % raccourci pour le support d'une densité
\def\trace{\text{tr}} % raccourci pour la trace d'une matrice 
\def\diag{\text{diag}} % raccourci pour le vecteur diagonal d'une matrice 

\def\grad{\nabla} % raccourci pour le gradient 

\def\GEV{{\cal{GEV}}} % raccourci pour la dénomination de la loi GEV
\def\GPD{{\cal{GPD}}} % raccourci pour la dénomination de la loi GPD
\def\EXPO{{\cal{E}}} % raccourci pour la dénomination de la loi exponentielle
\def\GAUSS{{\cal{N}}} % raccourci pour la dénomination de la loi gaussienne
\def\GEV{{\cal{GEV}}} % raccourci pour la dénomination de la loi GEV
\def\BERN{{\cal{B}}_e} % raccourci pour la dénomination de la loi de Bernoulli
\def\BINOM{{\cal{B}}} % raccourci pour la dénomination de la loi binomiale
\def\POIS{{\cal{P}}} % raccourci pour la dénomination de la loi de Poisson
\def\GAMMA{{\cal{G}}} % raccourci pour la dénomination de la loi gamma
\def\INVGAMMA{{\cal{IG}}} % raccourci pour la dénomination de la loi inverse gamma
\def\FS{{\cal{FS}}} % raccourci pour la dénomination de la loi de Fisher-Snedecor
\def\STU{{\cal{S}}_t} % raccourci pour la dénomination de la loi de Student

\def\iid{\textit{iid} } % raccourci pour le terme "i.i.d."
\def\va{\textit{va} } % raccourci pour le terme "variable aléatoire"
\def\EMV{$\text{EMV}$} % raccourci pour le terme "estimateur du maximum de vraisemblance"
\def\EMC{$\text{EMC}$} % raccourci pour le terme "estimateur des moindres carrés"




\newcounter{cptpropo}[part]
\newenvironment{propo}[0]
{\noindent\textsc{Proposition}\,\refstepcounter{cptpropo}\thecptpropo.\it}

\newcounter{cptlemmo}[part]
\newenvironment{lemmo}[0]
{\noindent\textsc{Lemme}\,\refstepcounter{cptlemmo}\thecptlemmo.\it}

\newcounter{cptexo}[part]
\newenvironment{exo}[0]
{\fbox{\noindent\textsc{Exemple}\,\refstepcounter{cptexo}\thecptexo.}\it $\ {}^{} \ $  }

%\newtheorem{theorem}{Théorème}
\usepackage{array,longtable}
\newtheorem{definition}{Définition}
\newtheorem{proposition}{Proposition}
%\newtheorem{proof}{Preuve}
\renewcommand{\theproof}{\empty{}} 
\newtheorem{lemma}{Lemme}


\newshadetheorem{theorem}{Théorème}

\newenvironment{proof2}
    {
    \begin{center}
    \begin{tabular}{|p{0.9\textwidth}|}
    \hline\\
    {\bf Preuve.} $\ {}^{} \ $} 
    { 
    \\\\\hline
    \end{tabular} 
    \end{center}
    \hspace{0.5cm}
    }

\newenvironment{proof}
    {
    {\bf \textcolor{blue}{Preuve.}} $\ {}^{} \ $} 

    

\newenvironment{rep2}
    {
      \begin{center}
    \begin{longtable}{|p{0.9\textwidth}|}
    \hline\\
    {\bf Réponse.} $\ {}^{} \ $} 
    { 
    \\\\\hline
    \end{longtable} 
    \end{center}
    \hspace{0.5cm}
    }
    
\newenvironment{rep}
    {
    {\bf \textcolor{purple}{Réponse}.} $\ {}^{} \ $} 
    

%\newtheorem{rep}[theorem]{\it R\'eponse}
\newtheorem{corollary}{Corollaire}
\newtheorem{exec}{\textcolor{blue}{Exercice}}
\newtheorem{assumption}{\noindent Hypothèse}
\newtheorem{acknowledgments}{\noindent Acknowledgments}
\newtheorem{example}{\noindent Exemple}
\newtheorem{lemma}{\noindent Lemme}



\newtheorem{remark}{\noindent Remarque}

\usepackage{imakeidx}
\makeindex


\newcommand{\Exemple}[1]{\fbox{\begin{minipage}{\textwidth}\begin{exo}#1\end{exo}\end{minipage}