\begin{proof}%[Preuve] % Espérance a priori
On a ${\cal{D}}=\Theta\in\R^d$ (ou plus généralement un espace de Hilbert) et $L(\theta,\delta)=\|\theta-\delta\|^2$ (norme euclidienne au carré). Par simplificité travaillons sur $\Theta=\R$ (d=1). Alors
\begin{eqnarray*}
R_{P}\left(\delta|\pi\right) & = & \int_{\Theta} (\theta-\delta)^2 \pi(\theta|x) \ d\theta, \\
& = & \E[\theta^2|x] - 2\delta\E[\theta|x] + \delta^2.
\end{eqnarray*}
En dérivant en $\delta$, on obtient 
\begin{eqnarray*}
R'_{P}\left(\delta|\pi\right) & = & 2\delta-2\E[\theta|x]
\end{eqnarray*}
valant 0 en $\delta=\E[\theta|x]$. Or
\begin{eqnarray*}
R''_{P}\left(\delta|\pi\right) & = & 2 \ > \ 0
\end{eqnarray*}
donc  $\delta\to R_{B}\left(\delta|\pi\right)$ est convexe, ce qui signifie que la solution de $R'_{P}\left(\delta|\pi\right)=0$ est bien un minimisateur du risque. 
 \end{proof}