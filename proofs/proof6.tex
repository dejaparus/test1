\begin{proof}%[Preuve] % Admissibilité d'un estimateur bayésien
Soit un estimateur bayésien $\delta^{\pi}$ de risque $R$ fini. Pour $\delta_0\in{\cal{D}}$ tel que $\forall\theta\in\Theta$, $R(\theta,\delta)\leq R(\theta,\delta_0)$, on note
\begin{eqnarray*}
{\cal{A}}_0 & = & \left\{\theta\in\Theta, \ R(\theta,\delta) \leq R(\theta,\delta_0) \right\}.
\end{eqnarray*}
On a alors
\begin{eqnarray*}
\int_{\Theta} R(\theta,\delta_0) d\Pi(\theta) - \int_{\Theta} R(\theta,\delta^{\pi}) d\Pi(\theta)& = & \int_{{\cal{A}}_0} \left(R(\theta,\delta)-R(\theta,\delta_0)\right)  d\Pi(\theta) \ \leq\ 0 
\end{eqnarray*}
avec égalité si et seulement si $\pi({\cal{A}}_0)=0$. Or, comme $\delta^{\pi}$ est bayésien et le risque fini, $R(\theta,\delta_0)\geq R(\theta,\delta^{\pi})$. Donc l'intégrale ci-dessus est négative et positive, donc nulle, ce qui sous-entend qu'en effet $\pi({\cal{A}}_0)=0$ (on dit alors que $\delta^{\pi}$ est $\pi-$admissible). Supposons cependant que $\delta^{\pi}$ n'est pas admissible. On  déduit de la démarche précédente que  $\exists\delta_0$ tel que $\forall \theta$ tel que $R(\theta,\delta_0)\leq R(\theta,\delta^{\pi})$ et $\theta_0\in\Theta$ tel que $R(\theta,\delta_0) < R(\theta,\delta^{\pi})$. La fonction définie sur $\Theta$ par $\theta\to R(\theta,\delta_0)-R(\theta,\delta^{\pi})$ est continue par hypothèse. Donc il existe un voisinage  ouvert $V_0\subset\Theta$ de $\theta_0$ tel que $\forall \theta\in V_0$, $R(\theta,\delta_0)<R(\theta,\delta^{\pi})$. On a  $\pi({\cal{A}}_0)\geq \pi(V_0)$. Or $\pi$ est supposé strictement positive sur $\Theta$, donc $\pi(V_0)>0$. L'ensemble ${\cal{A}}_0$ est donc de mesure non nulle, ce qui contredit la première partie de la démonstration. En conclusion, $\delta^{\pi}$ est admissible. 
\end{proof}