\begin{proof}%[Preuve] % Fonction de coût L1
Considérons la situation générique où 
\begin{eqnarray}
L_{c_1,c_2}(\theta,\delta) & = & \left\{\begin{array}{ll} c_2(\theta-\delta) & \text{si $\theta>\delta$} \\ c_1(\delta-\theta) & \text{sinon} 
\end{array} \right. \label{cout.lin}
\end{eqnarray}
Le risque {\it a posteriori} s'écrit
\begin{eqnarray*}
R_P(\delta|\pi) & = & \int_{-\infty}^\delta c_1(\delta-\theta)\pi(\theta|x) \ d\theta +  \int_{\delta}^{\infty} c_2(\theta-\delta)\pi(\theta|x) \ d\theta.
\end{eqnarray*}
En raisonnant par intégration par parties (IPP), il vient :
\begin{eqnarray*}
R_P(\delta|\pi) & = & \left[c_2(\theta-\delta) \Pi(\theta|x)\right]^{\delta}_{-\infty} + c_2 \Pi(\theta<\delta|x) +   
\left[c_1(\delta-\theta) \Pi(\theta\right]^{\infty}_{\delta} - c_1\Pi(\theta>\delta|x), \\
& = & c_2 \Pi(\theta<\delta|x) + c_1\left(1-\Pi(\theta<\delta|x)\right)
\end{eqnarray*}
car $\lim_{\theta\to-\infty} \Pi(\theta\x) = \lim_{\theta\to\infty} \Pi(\theta\x) = 0$. 
Ce risque est minimum lorsque $R_P(\delta|\pi)=0$ soit lorsque
\begin{eqnarray*}
\Pi(\theta<\delta^{\pi}|x) & = & \frac{c_1}{c_1+c_2}
\end{eqnarray*}
 ce qui confère donc à l'estimateur $\delta^{\pi}$ le sens du quantile de seuil  $\frac{c_1}{c_1+c_2}$.
\end{proof}