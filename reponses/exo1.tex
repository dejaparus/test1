\begin{rep} %[Exercice \ref{exo1}]
La vraisemblance dans les deux cas est
\begin{eqnarray*}
\ell(\theta|x_1,x_2) & \propto & \exp\left\{-(\bar{x}-\theta)^2\right\}
\end{eqnarray*}
et qui devrait donc conduire à la m\^eme inférence sur $\theta$. Mais $g(x_1,x_2|\theta)$ est très différente de la première distribution (par exemple, l'espérance de $x_1-x_2$ n'est pas définie). Les estimateurs de $\theta$ auront donc des propriétés fréquentistes différentes s'ils ne dépendent pas que de $\bar{x}$ {({\bf ex}: estimateur des moments)}. En particulier, les régions de confiance pour $\theta$ peuvent différer fortement car $g$ possède des queues plus épaisses. 
\end{rep}