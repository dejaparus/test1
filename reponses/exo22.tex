\begin{rep}
Il est pertinent au vu de l'expertise disponible de : (a) modéliser l'incertitude sur $X$ par la loi {\it a priori} prédictive dans un premier calcul de tenue de cuve ; cela implique de calibrer le prior ; (b) se servir de l'amplitude de cette loi pour préciser un domaine ultrasonone où chercher des défauts (constitution d'un échantillon de \emph{retour d'expérience}) ; (c) mélanger les deux informations pour améliorer la connaissance de $X$ {\it a posteriori}.
\begin{enumerate}
    \item La solution au problème du maximum d'entropie sous la contrainte linéaire (et en utilisant $\pi_0(\theta)\propto 1$
\begin{eqnarray*}
\int_{\theta} \theta \pi(\theta) \ d\theta & = & \theta_e
\end{eqnarray*}
mène directement à la solution
\begin{eqnarray*}
\pi(\theta) & \propto & \exp(-\lambda_1\theta)
\end{eqnarray*}
et l'on reconnaît le terme général d'une loi exponentielle d'espérance $1/\lambda_1=\theta_e$. 
\item On suppose ici que $\sigma_e$ est un estimateur de la variance prédictive {\it a priori}
\begin{eqnarray*}
\sqrt{\Var[X]} & = & \sqrt{\E[\Var[X|\theta]] + \Var[\E[X|\theta]]}, \\
& = & \sqrt{\E[\theta^2] + \Var[\theta]}, \\
& = & \sqrt{2\E[\theta^2] - \E^2[\theta]} \ = \ \sqrt{2\E[\theta^2] -\theta^2_e}.
\end{eqnarray*}
On peut alors maximiser l'entropie sous les contraintes linéaires
\begin{eqnarray*}
\int_{\theta} \theta \pi(\theta) \ d\theta \ = \ \theta_e & \text{et} & \int_{\theta} \theta^2 \pi(\theta) \ d\theta \ = \ \frac{1}{2}(\sigma^2_e + \theta^2_e).
\end{eqnarray*}
La loi {\it a priori} obtenue est gaussienne :
\begin{eqnarray*}
\theta & \sim & {\cal{N}}\left(\theta_e, \frac{1}{2}(\sigma^2_e - \theta^2_e)\right)
\end{eqnarray*}
et une condition de cohérence de l'expertise vis-à-vis de la modélisation paramétrique $X|\theta\sim f(x|\theta)$ est d'avoir $\sigma_e>\theta_e$.
\end{enumerate}
\end{rep}