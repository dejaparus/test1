\subsection{TP : Création d'un système d'alerte pour la circulation routière}

On s'intéresse à un évènement routier $X=x$ relevé par un système de détection vivant dans l'espace $\chi$ de dimension finie. Ce système de détection peut prédire des évènements répétés du type "un animal sur la voie", "accrochage", "accident", "bouchon"... La question est de déterminer si, à chaque fois qu'un événement routier $x$ est collecté, il est utile qu'une intervention de secours soit menée. \\

Nommons $\theta$ une variable indiquant la gravité de l'évènement. Cette variable a des valeurs dans les ensembles disjoints $\Theta_0$ (incidents sans gravité) et $\Theta_1$   (accidents nécessitant possiblement une intervention). On suppose disposer d'un échantillon labélisé  ${\bf e_n}=({\bf x_n},{\bf \theta_n})$. \\

\paragraph{\bf Questions.} \\
\begin{enumerate}
\item Lorsqu'une observation $x$ apparaît, comment prévoir $\theta$ ?
\item Comment peut-on en déduire une alarme efficace ? 
\end{enumerate}

\if\mycmdtpone1 \paragraph{\bf Réponses.} \\
{\bf 1.} Dans un cadre bayésien, probabilisons l'espace $\Theta= \Theta_0 \oplus \Theta_1$ et nommons $\Pi$ la mesure de probabilité associée. On note également $P$ la mesure de probabilité associée à l'espace $\chi$ supposé probabilisé. On souhaite prévoir la probabilité que $\theta\in\Theta_0$ sachant $x$ et $e_n$. {\it Via} une règle de Bayes, on produit le classifieur classique (dit {\it classifieur de Bayes})
\begin{eqnarray*}
\Pi(\theta\in\Theta_0|x,e_n) & = & \frac{P(X=x|\theta\in\Theta_0,e_n)\Pi(\theta\in\Theta_0|e_n)}{P(X=x|e_n)}, \\
& = & \frac{P(X=x|\theta\in\Theta_0,e_n)\Pi(\theta\in\Theta_0|e_n)}{P(X=x|\theta\in\Theta_0)\Pi(\theta\in\Theta_0|e_n) + P(X=x|\theta\notin\Theta_0)\left[1-\Pi(\theta\in\Theta_0|e_n)\right]}
\end{eqnarray*} 
Le terme de vraisemblance $P(X=x|\theta\in\Theta_0,e_n)$ peut être estimé de nombreuses manières (ex : par  la fréquence d'observation de l' évènement $X=x$ dans les situations recensées pour lesquelles $\theta\in\Theta_0$). Le classifieur de Bayes {\it naïf} repose ainsi sur une simplification de cette vraisemblance, etc. (voir cours d'apprentissage statistique). \\

\vspace{1cm}

{\bf 2.}  Le calcul de la probabilité $\Pi(\theta\in\Theta_0|x,e_n)$ ne suffit cependant pas pour prendre une décision opérationnelle. Intuitivement, on souhaiterait pourtant choisir de mener une intervention si
\begin{eqnarray*}
\Pi(\theta\in\Theta_0|X=x,e_n) \ \geq \  \Pi(\theta\notin\Theta_0|X=x,e_n) & = & \Pi(\theta\in\Theta_1|X=x,e_n), \\
& = & 1 - \Pi(\theta\in\Theta_0|X=x,e_n)
\end{eqnarray*}
et donc à  choisir (ou recommander) d'intervenir si
\begin{eqnarray}
\Pi(\theta\in\Theta_0|X=x,e_n) \geq 1/2. \label{regle.simpliste}
\end{eqnarray}
Mais cette règle est en fait simpliste, car elle n'intègre pas les risques d'erreur liés au fait qu'on utilise un échantillon de taille finie $n$ pour mener le calcul de cette probabilité. Une bonne fa\c con de faire est de placer le problème de classification dans un problème de décision plus vaste. \\

La décision qu'un possible intervenant sur le réseau routier souhaite prendre est binaire : sachant $X=x$, on intervient ou non. \`A partir des données $e_n$, il tente donc de définir un estimateur statistique $\hat{\delta}_n(x)$ d'une décision {\it idéale} $\delta(x)$ vivant dans un espace de décision ${\cal{D}}=\{0,1\}$ où : \\

\begin{itemize}
\item[$\bullet$] $\delta(x) = 0$ $\Leftrightarrow$ pas d'intervention, 
\item[$\bullet$] $\delta(x)=1 $ $\Leftrightarrow$ intervention. \\
\end{itemize}


Pourquoi parle-t-on d'estimateur statistique ? Parce que la décision idéale $\delta(x)$ est inaccessible par nature -- elle sous-entend que le possible intervenant est omniscient, que $\theta$ est parfaitement connu et que nécessairement $n=\infty$. \\

 On construit tout estimateur statistique comme le minimiseur d'une {\it fonction de coût} $$\delta(x)\in{\cal{D}}\mapsto L(\theta,\delta(x))$$ que l'on cherche à  définir si la vérité sur $\theta$ pouvait être connue. Dans le cas qui nous intéresse, on aurait : \\
 
\begin{itemize}
\item[$\bullet$] $L(\theta,\delta(x))=C_1 =$ le coût prévisionnel d'une intervention à  raison, donc si $\theta\in\Theta_1$ (ou $\theta\notin\Theta_0$) et $\delta(x)=1$ ; \\

\item[$\bullet$] $L(\theta,\delta(x))=C_2 = $ le coût prévisionnel d'une non-intervention à  tort ({\it erreur de 1ère espèce}),  si $\theta\in\Theta_1$ et $\delta(x)=0$ ; \\

\item[$\bullet$] $L(\theta,\delta(x))=C_3 = $ le coût prévisionnel d'une intervention à  tort ({\it erreur de 2ème espèce}), si $\theta\in\Theta_0$ et $\delta(x)=1$ ; \\

\item[$\bullet$] $L(\theta,\delta(x))=C_4 = 0 $ le coût (nul) d'une non-intervention à  raison, si  $\theta\in\Theta_0$ et $\delta(x)=0$. \\
\end{itemize}

On peut alors écrire, de fa\c con plus condensée :
\begin{eqnarray}
L(\theta,\delta(x)) & = & %\left[C_1\delta(x) + C_2(1-\delta(x))\right]\1_{\{\theta\in\Theta_0\}} + C_3\delta(x)\1_{\{\theta\notin\Theta_0\}}.
C_1\delta(x)\1_{\{\theta\in\Theta_1\}}  + C_2(1-\delta(x))\1_{\{\theta\in\Theta_1\}} + C_3\delta(x)\1_{\{\theta\in\Theta_0\}}. \label{f.cout}
\end{eqnarray}
Or la vérité sur $\theta$ ne peut être parfaitement connue. On dispose simplement de la connaissance {\it a posteriori} $\Pi(\theta\in\Theta|X=x,e_n)$. Si l'on souhaite prendre une décision qui prenne en compte l'incertitude épistémique sur $\theta$, il faut définir un {\it risque} $R(\delta(x),\Pi,e_n)$ qui puisse intégrer la connaissance de $\Pi(\theta\in\Theta|X=x,e_n)$, la construction de $L(\theta,\delta(x))$ et qui soit minimisable en un choix unique d'estimateur de $\delta(x)$. Pour obtenir un {\it ordre total} sur l'espace des applications $\delta(x)\mapsto R(\delta(x),\Pi,e_n)$, il faut nécessairement définir ce risque comme le {\it risque de Bayes} 
\begin{eqnarray*}
R(\delta(x),\Pi,e_n) & = & \int_{\Theta} L(\theta,\delta(x)) d\Pi(\theta\in\Theta|X=x,e_n)
\end{eqnarray*}
et d'en déduire donc la {\it décision optimale} (et non pas {\it idéale})
\begin{eqnarray*}
\hat{\delta}_n(x) & = & \arg\min\limits_{\delta(x)\in{\cal{D}}} R(\delta(x),\Pi,e_n).
\end{eqnarray*}       
%\textcolor{blue}{A vous de produire le résultat ! Comment calibrer le classifieur ?} 

Après intégration,
\begin{eqnarray*}
R(\delta(x),\Pi,e_n) & = & C_1\delta(x)\Pi(\theta\in\Theta_1|X=x,e_n) + C_2(1-\delta(x))\Pi(\theta\in\Theta_1|X=x,e_n)  \\
& & \ + C_3\delta(x)\left[1-\Pi(\theta\in\Theta_1|X=x,e_n)\right].
\end{eqnarray*}
On en déduit la règle de décision (ou de recommandation) suivante : ayant observé l'évènement $X=x$, on décide d'intervenir ($\hat{\delta}_n(x)=1$) si le risque associé à cette décision est moins élevé que le risque associé à la décision contraire ($\hat{\delta}_n(x)=0$), soit si 
\begin{eqnarray*}
R(0,\Pi,e_n) \ = \ C_2 \Pi(\theta\in\Theta_1|X=x,e_n)  &  \geq & R(1,\Pi,e_n) \ = \ C_1\Pi(\theta\in\Theta_1|X=x,e_n) \\
& & \hspace{2cm} + C_3\left[1-\Pi(\theta\in\Theta_1|X=x,e_n)\right],
\end{eqnarray*}
c'est-à-dire quand 
\begin{eqnarray}
\Pi(\theta\in\Theta_1|X=x,e_n) & \geq & \frac{C_3}{C_2-C_1+C_3} \label{bonne.regle}
\end{eqnarray}

%et pour $\delta(x)\in{\cal{D}}=\{0,1\}$, il vient (en notant que $\Pi(\theta\in\Theta_1|X=x,e_n)=1-\Pi(\theta\in\Theta_0|X=x,e_n)$)
%\begin{eqnarray*}
%R(0,\Pi,e_n) \ \geq \ R(1,\Pi,e_n) & \Leftrightarrow & \Pi(\theta\in\Theta_0|X=x,e_n) \ \geq \ \frac{C_3}{C_2-C_1+C_3}.
%\end{eqnarray*}


\vspace{1cm}

Finissons par quelques remarques importantes :
\begin{itemize}
\item[$\bullet$] il n'est pas utile de disposer des coûts absolus $C_i$ pour prendre une décision, car des rapports de coûts suffisent, ce qui est en général plus simple à  estimer dans une vraie démarche opérationnelle ;
\item[$\bullet$] on peut légitimement supposer que $C_1\leq  C_3 < C_2$, car le coût $C_1$ d'une intervention utile peut être réduit par l'effet d'assurances, tandis que le coût $C_3$ d'une intervention inutile ne l'est pas. Enfin, le coût (prévisionnel) $C_2$ d'une non-intervention qui aurait pu être utile peut éventuellement intégrer celui de vies humaines ; remarquons qu'en toute rigueur, $C_2=C_2(t)$ où $t$ représente le temps depuis l'occurence de l'événement, et que cette fonction est très certainement croissante. 
\item[$\bullet$] Il faut que $C_3=C_2$ et $C_1=0$ pour obtenir l'équivalent décisionnel de la règle (\ref{regle.simpliste}). On per\c coit bien qu'une décision purement intuitive est à  rejeter, car elle pré-suppose une contradiction forte avec la deuxième remarque.
\item[$\bullet$] Cette règle de décision prend intégralement en compte les incertitudes sur la véritable nature de l'événement $\theta$, conditionnellement à  la validité des hypothèses. 
\item[$\bullet$] Le choix de la fonction de coût (\ref{f.cout}) est arbitraire ; mais retenons qu'en l'absence d'arguments permettant de rationaliser ce choix, les coûts sont généralement assemblés de fa\c con additive
\end{itemize}  
\fi


