\begin{abstract}
Ce cours a pour objectif de présenter d’une part les principales méthodologies de modélisation bayésienne appliquées à des problèmes d’aide à la décision en univers risqué, 
et d’autre part les principales méthodes de calcul inférentiel permettant l’enrichissement de l’information utile, en fonction de l’emploi et de la nature des modèles. Il nécessite les pré-requis suivants : notions fondamentales de probabilités et statistique, introduction aux statistiques bayésiennes, méthodes de Monte-Carlo, calcul scientifique en \texttt{R} ou/ et en \texttt{Python}. Tout au long du cours, des liens avec l'apprentissage statistique (\textit{machine learning}) sont présentés.\\

Ce document évolue au fil du temps, et comporte parfois des coquilles ou quelques contresens qui peuvent m'échapper. Certains de mes étudiants, par leurs remarques et parfois leur aide pour déceler ces coquilles, par leurs demandes d'éclaircissement, contribuent notablement à améliorer son propos et sa fluidité. Qu'ils en soient particulièrement remerciés, et plus spécialement Paul Liautaud (M2A 2022).  Enfin, ce document fait parfois quelques emprunts graphiques à des ouvrages par ailleurs recommandés. 

%Tout en évoluant au fil du temps, il considère plus spécifiquement les méthodes et outils (théoriques et pratiques) suivants :
%\begin{itemize}
%\item Formalisation et résolution de problèmes d’aide à la décision en univers risqué
%\item Représentation probabiliste des incertitudes (Cox-Jaynes, de Finetti)
%\item Maximum d’entropie, familles exponentielles, modélisation par données virtuelles
%\item Modèles hiérarchiques
%\item Règles d’invariance, de compatibilité et de cohérence pour les modèles bayésiens
%\item Méthodes d'échantillonnage (rejet, importance, Gibbs, MCMC, MCMC adaptatives, méthodes de filtrage 
%\item Quelques perspectives : quadrature bayésienne, modèles hiérarchiques de haute dimension, modélisation bayésienne fonctionnelle, processus gaussiens, calibration par expériences numériques, critères d’enrichissement bayésiens \\
%\end{itemize}



\end{abstract}