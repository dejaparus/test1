\section{Propriétés fondamentales du cadre bayésien}\label{proprietes}

\subsection{Prédiction (prévision)}\label{predictivite}

Le contexte du problème de la prédiction est le suivant : les observations X sont identiquement distribuées selon $P_{\theta}$, qui est absolument continue par rapport à une mesure dominante $\mu$. Il existe donc une fonction de densité conditionnelle $f(·|\theta)$. Par ailleurs on suppose que $\theta$ suit une loi a priori $\pi$. \emph{Mener une prévision} consiste alors, 
à partir de $n$ tirages observés $x_1,\ldots,x_n$, de déterminer le plus précisément possible ce que pourrait être le tirage suivant $X_n+1$. \\

Dans l’approche fréquentiste, on calcule dans les faits $f(x_{n+1}|x_1,··· ,x_n,\hat{\theta}_n)$, puisqu'on ne connaît pas $\theta$ et qu'on doit l'estimer : on utilise donc deux fois les données (une fois pour l'estimation de $\theta$, et une nouvelle fois pour la prévision). En règle générale, ceci amène   à sous-estimer les intervalles de confiance. \\

La stratégie du paradigme bayésien consiste à intégrer la prévision suivant la loi courante {\it a posteriori} sur $\theta$ et ce, afin d’avoir la meilleure prévision compte-tenu à la fois de notre savoir et de notre ignorance sur le paramètre. La loi prédictive s’écrit ainsi :
\begin{eqnarray*}
f(X_{n+1}|x_1,\ldots,x_n) & = & \int_{\Theta} f(X_{n+1}|x_1,\ldots,x_n,\theta) \pi(\theta|x_1,\ldots,x_n) \ d\theta
\end{eqnarray*}
qui s'écrit plus simplement, lorsque \emph{sachant $\theta$} les tirages sont iid :
\begin{eqnarray*}
f(x_{n+1}|x_1,\ldots,x_n) & = & \int_{\Theta} f(x_{n+1}|\theta) \pi(\theta|x_1,\ldots,x_n) \ d\theta.
\end{eqnarray*}
Ainsi, le prédicteur de $X_{n+1}$ sous le coût quadratique est
\begin{eqnarray*}
\E[X_{n+1}|x_1,\ldots,x_n] & = & \int_{\Omega} x f(x|x_1,\ldots,x_n) \ dx. 
\end{eqnarray*}

%\begin{exec}{Régression gaussienne}
%Supposons avoir $n$ v.a. iid  $Y_1,\ldots,Y_n\sim{\cal{N}}(\mu,\sigma^2)$ et choisissons $\pi(\mu,\sigma)\propto 1/\sigma$ (mesure de Haar). 
%\end{exec}


\subsection{Propriétés asymptotiques}\label{asymptotique}

Les approches classique et bayésienne de la modélisation et de la décision statistique aboutissent à des résultats similaires à l'asymptotisme, et les principaux théorèmes classiques connaissent leur pendant bayésien. Ainsi, le théorème central limite "classique" devient le théorème de Bernstein-von Mises dans le cadre bayésien (on l'appelle également \emph{théorème central limite bayésien} par abus de langage). Afin de comparer les deux approches, on doit d'abord définir ce que signifie "vraie valeur $\theta_0$ du paramètre $\theta$". \\

%Le théorème de De Finetti nous indique que  la loi jointe des données $x_1,\ldots,x_n$ considérées comme échangeables et indépendantes, pour $n\to\infty$, est la prédictive {\it a priori}
%\begin{eqnarray*}
%m_{\pi}(x_1,\ldots,x_n) & = & \int_{\Theta} f(x_1,\ldots,x_n|\theta) \pi(\theta) \ d\theta.
%\end{eqnarray*}
%Cette loi est donc ce qui se rapproche le plus de 
Notons $\tilde{f}(x)$ la "vraie loi" inconnue des données.  Si on fait maintenant le choix d'une loi paramétrique $X\sim f(x|\theta_{0})$ (ou mécanisme génératif), alors la loi $f(x|\theta_0)$ doit être la plus proche possible de $\tilde{f}(x)$. Cette notion de proximité est généralement définie de la fa\c con suivante.

\begin{definition}
Soit $\tilde{f}(x)$ la loi inconnue des données. On  définit $\theta_0$ par
\begin{eqnarray*}
\theta_0 & = & \arg\min\limits_{\theta\in\Theta}  KL\left(\tilde{f}(x) || f(x|\theta)\right)
\end{eqnarray*}
où KL est la divergence de Kullback-Leibler. On notera par la suite plus simplement ce terme $KL(\theta)$.
\end{definition}

\begin{theorem}{Consistance}\label{consistance.post}
Si $f(.|\theta)$ est suffisamment régulière et \textcolor{black}{identifiable}, soit si $\theta_1\neq\theta_2 \Rightarrow f(x|\theta_1)\neq f(x|\theta_2)$ $\forall x\in\Omega$, alors pour tout échantillon $\bf x_n$ iid
\begin{eqnarray*}
\pi(\theta|{\bf x_n}) & \xrightarrow{p.s.}{} & \delta_{\theta_0}.
\end{eqnarray*}
Par ailleurs, si $g:\Theta\to\R$ est mesurable et telle que $\E[g(\theta)]<\infty$, alors sous les mêmes hypothèses
\begin{eqnarray*}
\lim\limits_{n\to\infty} \E\left[g(\theta)|X_1,\ldots,X_n\right] & = & g(\theta) \ \ \ \text{p.s.}
\end{eqnarray*}
\end{theorem}

Un résultat utile, intermédiaire entre la consistance et la convergence en loi (Théorème \ref{von.mises}), est la convergence en probabilité.  

\begin{theorem}
Si $\Theta$ est fini et discret et $\Pi(\theta=\theta_0)>0$, alors pour tout échantillon iid $X_1,\ldots,X_n|\theta\sim f(X|\theta)$,
\begin{eqnarray*}
\Pi\left(\theta=\theta_0|X_1,\ldots,X_n\right) & \xrightarrow[\Pp]{n\to\infty} & 1.
\end{eqnarray*}
\end{theorem}

\if\mycmdprooftwo1 \begin{proof}%[Preuve] % Convergence en probabilité (cas discret)
On va montrer que $\Pi(\theta|X_1,\ldots,X_n)\to 0$ $\forall \theta\neq\theta_0$. On a
\begin{eqnarray}
\log \frac{\Pi(\theta|X_1,\ldots,X_n)}{\Pi(\theta_0|X_1,\ldots,X_n)} & = & \log \frac{\Pi(\theta)}{\Pi(\theta_0)} + \sum\limits_{i=1} Y_i \label{decomp1}
\end{eqnarray}
où les 
$$
Y_i = \log \frac{f(X_i|\theta)}{f(X_i|\theta_0)}
$$
 sont des v.a. iid, telles ques
 \begin{eqnarray*}
 \E\left[Y_i\right] & = & KL\left(\theta_0\right) - KL\left(\theta\right)
 \end{eqnarray*}
 {\it (rappelons que l'intégration se fait par rapport à la vraie loi inconnue des données)}
 qui vaut 0 si $\theta=\theta_0$, et qui est négatif sinon car $\theta_0$ est l'unique minimiseur de $KL(\theta)$. Ainsi, si $\theta\neq\theta_0$, le second terme de (\ref{decomp1}) est une somme de termes iid avec une espérance négative. Par la loi des grands nombres, on obtient que $\lim_{n\to\infty} \sum_{i=1} Y_i = -\infty$ si $\theta\neq\theta_0$. Tant que le premier terme de (\ref{decomp1}) est fini (soit tant que $\Pi(\theta=\theta_0)>0$), l'expression totale pour (\ref{decomp1}) tend également vers $-\infty$. Nécessairement, 
 \begin{eqnarray*}
 \frac{\Pi(\theta|X_1,\ldots,X_n)}{\Pi(\theta_0|X_1,\ldots,X_n)} & \xrightarrow[\Pp]{n\to\infty} 0
 \end{eqnarray*}
 et donc $\Pi(\theta|X_1,\ldots,X_n)\to 0$ $\forall \theta\neq\theta_0$. Comme toutes les probabilités somment à 1, nécessairement \begin{eqnarray*}
\Pi\left(\theta=\theta_0|X_1,\ldots,X_n\right) & \xrightarrow[\Pp]{n\to\infty} & 1.
\end{eqnarray*}
\end{proof}
\fi
\vspace{1cm}

Si $\Theta$ est continu, alors $\pi(\theta_0|x)$ vaut toujours 0 pour tout échantillon fini $x$, et on ne peut appliquer les outils menant au résultat précédent. Pour adapter cette preuve, il faut définir un voisinage $V_{\theta_0}$ qui est un ensemble ouvert de points de $\Theta$ à une distance maximum fixée de $\theta_0$ ($\Theta$ étant un espace métrique). 

\begin{theorem}
Si $\Theta$ est un ensemble compact et si $V_{\theta_0}$ est tel que $\Pi(\theta\in V_{\theta_0})>0$ avec
\begin{eqnarray*}
\theta_0 & = & \arg\min\limits_{\theta\in\Theta} KL(\theta)
\end{eqnarray*}
alors 
\begin{eqnarray*}
\Pi(\theta\in V_{\theta_0}|x_1,\ldots,x_n) & \xrightarrow[\Pp]{n\to\infty} & 1.
\end{eqnarray*}
\end{theorem}

%\if\mycmdproof1 %\begin{proof}%[Preuve] % Convergence en probabilité (cas continu)
On admet ce résultat.
\end{proof}


%\fi
%\vspace{1cm}


Le théorème de Bernstein-von Mises suppose l'existence de l'information de Fisher $I_{\theta}$. Il n'existe pas d'ensemble de conditions de régularité minimal nécessaire pour l'existence de $I_{\theta}$ ; cependant, la plupart des auteurs s'accordent sur les conditions suffisantes suivantes d'existence, de positivité et de continuité dans un sous-espace de $\Theta$ :
\begin{itemize}
    \item $f(x|\theta)$ est  absolument continue en $\theta$ ;
    \item sa dérivée doit exister pour tout $x\in\Omega$.
\end{itemize}
Alors
\begin{eqnarray*}
I_{\theta} & = & \E\left[\left(\frac{\partial }{\partial \theta} \log f(X|\theta)\right)^2\right] \ 
 = \ - \E\left[\frac{\partial^2 }{\partial \theta^2} \log f(X|\theta)\right]
\end{eqnarray*}
si $\log f(x|\theta)$ est deux fois différentiable en $\theta$. 


\begin{theorem}{Normalité asymptotique (\textbf{Bernstein-von Mises})}\label{von.mises}
Soit $I_{\theta}$ la matrice d'information de Fisher du modèle $f(.|\theta)$ et soit $g(\theta)$ la densité de la gaussienne ${\cal{N}}(0,I^{-1}_{\theta_0})$. Soit $\hat{\theta}_n$ le maximum de vraisemblance. Alors, dans les conditions précédentes,
\begin{eqnarray*}
\int_{\Theta}\left|\pi\left(\sqrt{n}\left\{\theta - \hat{\theta}_n\right\}|{\bf x_n}\right) - g(\theta)\right| \ d\theta & \rightarrow & 0.
\end{eqnarray*}
\end{theorem}

\if\mycmdprooftwo1 \begin{proof}%[Preuve] % Théorème de Bernstein von Mises
Le théorème \ref{consistance.post} montre qu'on peut concentrer l'étude sur un voisinage de $\theta_0$. Obtenir la loi limite requiert deux étapes :
\begin{itemize}
    \item montrer que le mode {\it a posteriori} est consistant, c'est-à-dire qu'il se situe dans le voisinage de $\theta_0$ où se situe presque toute la masse ;
    \item montrer l'approximation gaussienne centrée en le mode {\it a posteriori}. 
\end{itemize}
Pour simplifier, le schéma de preuve donné ici considère que $\theta$ est un scalaire. Notons $\tilde{\theta}_n$ le mode {\it a posteriori}
$$
\tilde{\theta}_n = \arg\max\limits_{\theta\in\Theta} \left\{\log f(x_1,\ldots,x_n|\theta) + \log \pi(\theta)\right\}.
$$
La preuve de consistance du MLE peut être adaptée pour montrer que $\tilde{\theta}_n\to\theta_0$ quand $n\to\infty$, presque sûrement. On peut alors approximer la log-densité {\it a posteriori} par un développement de Taylor centré autour de $\tilde{\theta}_n$ (approximation quadratique de $\log \pi(\theta|x_1,\ldots,x_n)$) :
\begin{eqnarray}
\log \pi(\theta|x_1,\ldots,x_n) & = & \log \pi(\tilde{\theta}_n|x_1,\ldots,x_n) + \frac{1}{3}(\theta-\tilde{\theta}_n)^2 \frac{\partial^2}{\partial \theta^2}\left[\log \pi(\theta|x_1,\ldots,x_n)\right]_{\theta=\tilde{\theta}_n} \\
& & \ \ + \  \frac{1}{6} (\theta-\tilde{\theta}_n)^3 \frac{\partial^3}{\partial \theta^3}\left[\log \pi(\theta|x_1,\ldots,x_n)\right]_{\theta=\tilde{\theta}_n} + \ldots \label{deploiement}
\end{eqnarray}
Le terme linéaire est nul car par définition du mode :
\begin{eqnarray*}
\frac{\partial }{\partial \theta}\left[\log \pi(\theta|x_1,\ldots,x_n)\right]_ {\theta=\tilde{\theta}_n} & = & 0.
\end{eqnarray*}
Le premier terme de (\ref{deploiement}) est constant. Le coefficient du second terme (sous l'hypothèse iid) est 
\begin{eqnarray*}
\frac{\partial^2}{\partial \theta^2}\left[\log \pi(\theta|x_1,\ldots,x_n)\right]_{\theta=\tilde{\theta}_n} & = & \frac{\partial^2}{\partial \theta^2}\left[\log \pi(\theta)\right]_{\theta=\tilde{\theta}_n} + \sum\limits_{i=1}^n \frac{\partial^2}{\partial \theta^2} \left[\log f(x_i|\theta)\right]_{\theta=\tilde{\theta}_n}, \\
& = & cte + \sum\limits_{i=1}^n Y_i
\end{eqnarray*}
où les $Y_i$ sont des variables aléatoires iid d'espérance négative sous l'hypothèse $X\sim \tilde{f}(x)$. En effet,  si $\tilde{\theta}_n=\theta_0$, on a 
\begin{eqnarray*}
\E\left[Y_i\right] & = & \E\left[\frac{\partial^2}{\partial \theta^2} \log f(x_i|\theta_0)\right] \ = \ - I_{\theta_0}
\end{eqnarray*}
et sinon
\begin{eqnarray*}
\E\left[Y_i\right] & = & - -\frac{\partial^2}{\partial \theta^2} KL\left(\theta\right)_{\theta=\tilde{\theta}_n} \ < \ 0 
\end{eqnarray*}
par convexité. Ainsi, le coefficient du second terme converge vers $-\infty$ à la vitesse $n$. Quand $n\to\infty$, $|\tilde{\theta}_n-\theta_0|\to 0$, et les termes suivants du développement de Taylor tendent vers 0. On a donc
\begin{eqnarray*}
\log \pi(\theta|x_1,\ldots,x_n) & \sim & -\alpha (\theta-\tilde{\theta}_n)^2
\end{eqnarray*}
et la loi limite de $\pi(\theta|x_1,\ldots,x_n)$ est donc une gaussienne.
\end{proof}
\fi
\vspace{1cm}

%%%%%%%%%%%%%%%%%%%%%%%%%%%%%%%
\subsection{Régions de crédibilité et régions HPD}\label{crédibilité}

Soit $x\sim f(.|\theta)$ une (ou plusieurs) observations. 

\begin{definition}{Région $\alpha-$crédible}
Une région $A$ de $\Theta$ est dite \textcolor{black}{$\alpha-$crédible} si $\Pi(\theta\in A|x)\geq 1-\alpha$.
\end{definition}

Notons que le paradigme bayésien permet une nouvelle fois de s’affranchir d’un inconvénient de l’approche fréquentiste.
Rappelons qu'au sens fréquentiste, $A$ est une \textcolor{black}{région de confiance $1-\alpha$} si, en refaisant l'expérience (l'observation d'un $X\sim f(.|\theta)$) un nombre de fois tendant vers $\infty$,
\begin{eqnarray*}
P_{\theta}(\theta\in A) & \geq & 1-\alpha.
\end{eqnarray*}
% En refaisant l’expérience un grand nombre de fois, la probabilité que $\theta$ soit dans $A$ est plus grande que $1 − \alpha$. 
Une région de confiance n’a donc de sens que pour un très grand nombre d’expériences tandis que la définition bayésienne exprime que la probabilité que $\theta$ soit dans $A$ au vue des celles déjà réalisées est plus grande que $1-\alpha$. Il n’y a donc pas besoin ici d’avoir recours à un nombre infini d’expériences pour définir une région $\alpha$-crédible, seule compte l’expérience effectivement réalisée. \\
 
 



\begin{remark}
On distingue bien ici la probabilité "fréquentiste" $P_{\theta}$ de la probabilité bayésienne $\Pi$. Dans le premier cas, l'aléatoire concerne la région $A$, qui est un estimateur statistique dépendant d'un estimateur classique $\hat\theta(X_1,\ldots,X_n)$ et $\theta$ est considéré comme fixe. Dans le second cas, c'est bien $\theta$ qui est une aléatoire.
\end{remark}

Il y a une infinité de régions $\alpha-$crédibles, il est donc logique de s’intéresser
à la région qui a le volume minimal. Le volume étant défini par $\mbox{vol}(A) = \int_A d\mu(\theta)$, si $\pi(\theta|x)$ est absolument continue par rapport à une mesure de 
référence $\mu$.

\begin{definition}{\bf Région HPD.}
$A_{\alpha,\pi}$ est une région HPD (\textit{highest posterior density}) si et seulement si 
\begin{eqnarray*}
A_{\alpha,\pi} & = & \left\{\theta\in\Theta, \  \pi(\theta|x)\geq h_{\alpha}\right\}
\end{eqnarray*}
où $h_{\alpha}$ est défini par 
\begin{eqnarray*}
h_{\alpha} &  = & \sup_h\left\{ \Pi\left(\theta | \pi(\theta|x)\geq h, X\right) \geq 1-{\alpha}\right\}.
\end{eqnarray*}
\end{definition}

$A_{\alpha,\pi}$ est parmi les régions qui ont une probabilité supérieure à $1-{\alpha}$ de contenir $\theta$ (et qui sont donc $\alpha$-crédibles) et sur lesquelles la densité {\it a posteriori} ne descend pas sous un certain niveau (restant au dessus de la valeur la plus élevée possible). \\

\begin{theorem}
$A_{\alpha,\pi}$ est parmi les régions $\alpha$-crédibles celle de volume minimal si et seulement si elle est HPD.
\end{theorem}



\begin{exec}
Soit $x_1,\ldots,x_n$ des réalisations iid de loi ${\cal{N}}(\mu,\sigma^2)$. On choisit la mesure {\it a priori} (non probabiliste) jointe
\begin{eqnarray*}
\pi(\mu,\sigma^2) & \propto & 1/\sigma^2.
\end{eqnarray*}
\begin{enumerate}
\item Déterminez la loi {\it a posteriori} jointe $\pi(\mu,\sigma^2|x_1,\ldots,x_n)$ 
\item Déterminez la loi {\it a posteriori} marginale $\pi(\mu|x_1,\ldots,x_n)$ 
\item Calculez la région HPD de seuil $\alpha$ pour $\mu$ et comparez-la à la région de confiance fréquentiste, de même seuil, qu'on pourrait calculer par l'emploi du maximum de vraisemblance.
\item Déterminez la loi {\it a posteriori} marginale $\pi(\sigma^2|x_1,\ldots,x_n)$ ; le calcul de la région HPD est-il simple ?
\end{enumerate}
\paragraph{Remarque.} ``Déterminez" signifie indiquer si la loi appartient à une famille connue, par exemple largement implémentée sur machines. La connaissance des lois gamma, inverse gamma et Student est peut-être nécessaire pour répondre aux questions.
\end{exec}

\if\mycmdexo1 \begin{rep}% Régions HPD
\paragraph{1. Loi jointe {\it a posteriori}.}
On a 
\begin{eqnarray*}
\pi(\mu,\sigma^2|x_1,\ldots,x_n) & \propto & \sigma^{-(n+2)} \exp\left(-\sum\limits_{i=1}^n \frac{(x_i-\mu)^2}{2\sigma^2}\right), \\
& = & \sigma^{2(-(n/2+1)} \exp \left(-\sum\limits_{i=1}^n \frac{x^2_i}{2\sigma^2} +\frac{2\mu\sum\limits_{i=1}^n x_i}{2\sigma^2} - n\mu^2/2\sigma^2 \right), \\
& = & \sigma^{2(-(n/2+1)} \exp \left( - \frac{n\left(\mu-\bar{x}_n\right)^2}{2\sigma^2}
\right) \exp \left( -\frac{1}{2\sigma^2} S_n^2\right)
\end{eqnarray*}
avec
\begin{eqnarray*}
S^2_n & = & -\sum\limits_{i=1}^n \left(\bar{x}^2_n-x^2_i\right) \ = \  \sum\limits_{i=1}^n \left(\bar{x}_n-x_i\right)^2 \ \geq \ 0.
\end{eqnarray*}
\paragraph{2. Loi marginale {\it a posteriori} de $\mu$.}
On obtient alors,
%\begin{eqnarray*}
%\pi(\mu|\sigma^2,x_1,\ldots,x_n) & \propto & \exp\left(-\frac{n}{2\sigma^2}\left(\mu-\bar{x}_n\right)^2 \right) \ \equiv \ {\cal{N}}\left(\bar{x}_n, \sigma^2/n\right)
%\end{eqnarray*}
%et, 
par intégration,
\begin{eqnarray*}
\pi(\mu|x_1,\ldots,x_n) & = & \int_{\R^+} \pi(\mu,\sigma^2|x_1,\ldots,x_n) \ d\sigma^2, \\
& \propto & \int_{\R^+} \sigma^{2(-(n/2+1)} \exp \left( - \frac{n\left(\mu-\bar{x}_n\right)^2}{2\sigma^2} -\frac{1}{2\sigma^2} S_n^2
\right) \ d\sigma^2.
\end{eqnarray*}
En reconnaissant dans l'intégrande le terme général d'un loi inverse gamma ${\cal{IG}}(n/2,(S^2_n + n(\mu-\bar{x}_n)^2)/2)$, on a alors
\begin{eqnarray*}
\pi(\mu|x_1,\ldots,x_n) & \propto & \Gamma(1)\left(S^2_n + n(\mu-\bar{x}_n)^2\right)^{-n/2}, \\
& \propto & \left(1+k_n u^2\right)^{-(k-1)/2}
\end{eqnarray*}
avec $k=n+1$ et $u=\sqrt{k k_n}(\mu-\bar{x}_n)$ et $k_n=n/S^2_n$. On peut alors réécrire plus simplement
\begin{eqnarray*}
\pi(\mu|x_1,\ldots,x_n) & \propto & \left(1+u^2/k\right)^{-(k-1)/2}
\end{eqnarray*}
qui définit le terme général d'une loi de Student en $u$ à $k$ degrés de liberté. Posons alors formellement la variable 
$$
u = \frac{\sqrt{n(n+1)}}{S_n}(\mu-\bar{x}_n) \ = \ g(\mu)
$$
et procédons à changement de variable (cf. Proposition \ref{changement.var}) pour connaître la loi de $u$. On a 
\begin{eqnarray*}
g^{-1}(u) & = & u \frac{S_n}{\sqrt{n(n+1)}} + \bar{x}_n, \\
g'(u) & = &  \frac{\sqrt{n(n+1)}}{S_n}.
\end{eqnarray*}
Donc
\begin{eqnarray*}
\pi_U(u) & \propto & \pi_{\mu}(g^{-1}(u)), \\
& \propto & (1+u^2/k)^{-\frac{k+1}/2}.
\end{eqnarray*}
Donc la loi {\it posteriori} de la variable $u$ est bien une Student à $k=n+1$ degrés de liberté :
\begin{eqnarray*}
u|x_1,\ldots,x_n & = & \frac{\sqrt{n(n+1)}}{S_n}(\mu-\bar{x}_n) \ \sim \ {\cal{S}}_t(n+1).
\end{eqnarray*}
On dit alors que la loi de $\mu$ est une \emph{Student décentrée} à $k$ degrés de liberté. 

\paragraph{3. Région HPD pour $\mu$.}
On veut alors déterminer 
\begin{eqnarray*}
{\cal{A}}_{\alpha,\pi} & = & \left\{\mu, \ \pi(\mu|x_1,\ldots,x_n ) \geq 1-\alpha\right\}.
\end{eqnarray*}
Donc en notant $t_{n+1}(\alpha)$ le quantile de seuil $\alpha$ de la loi ${\cal{S}}_t(n+1)$, on a (par symmétrie de cette loi)
\begin{eqnarray*}
\Pi\left(-t_{n+1}(1-\alpha/2) \leq u \leq (t_{n+1}(1-\alpha/2)|x_1,\ldots,x_n\right) & = & 1-\alpha
\end{eqnarray*}
soit, de fa\ con équivalente,
\begin{eqnarray*}
\Pi\left(\bar{x}_n-\frac{S_n}{\sqrt{n(n+1}}t_{n+1}(1-\alpha/2) \leq \mu \leq \bar{x}_n+\frac{S_n}{\sqrt{n(n+1}}t_{n+1}(1-\alpha/2)|x_1,\ldots,x_n\right) & = & 1-\alpha.
\end{eqnarray*}
Rappelons que la région fréquentiste de seuil $1-\alpha$ connue pour l'estimation du maximum de vraisemblance (EMV), dès qu'on a pris conscience que l'EMV de $\sigma^2$ est
\begin{eqnarray*}
\hat{\sigma}^2_n & = & S^2_n/n,
\end{eqnarray*}
est :
\begin{eqnarray*}
\left[\bar{x}_n-\frac{S_n}{n}t_{n+1}(1-\alpha/2) \ ; \ \bar{x}_n+\frac{S_n}{n}t_{n+1}(1-\alpha/2)\right].
\end{eqnarray*}
Avec $n<\sqrt{n(n+)}$ on voit que la région bayésienne est légèrement plus grande que la région fréquentiste, mais qu'il y a équivalence asymptotique (ce qui est logique). 


\paragraph{4. Loi marginale {\it a posteriori} de $\sigma^2$.} De la même fa\c con que précédemment, on a 
\begin{eqnarray*}
\pi(\sigma^2|x_1,\ldots,x_n) & \propto & \frac{1}{\sigma^{n+2}}   \exp\left(-S^2_n/2\sigma^2\right) \int_{\R} \exp\left(-\frac{(\mu-\bar{x}_n)^2}{2\sigma^2}\right) \ d\mu, \\
 & \propto & \frac{1}{\sigma^{n+2}}   \exp\left(-S^2_n/2\sigma^2\right) \frac{\sqrt{2\pi}}{\sqrt{n}}\sigma, \\
 & \propto &  \frac{1}{\left(\sigma^2\right)^{n/2+1/2}} \exp\left(-S^2_n/2\sigma^2\right)
\end{eqnarray*}
et on reconnaît ici le terme général d'une loi inverse gamma :
\begin{eqnarray*}
\sigma^2|x_1,\ldots,x_n  & \sim & {\cal{IG}}(n/2-1/2,S^2_n/2)
\end{eqnarray*}
qui n'est définie que si $n>1$ (ce qui semble logique : il faut au moins avoir deux données pour inférer sur la variance $\sigma^2$). De plus, pour avoir $S^2_n\neq 0$ si $n=2$, il faut que $x_1\neq x_2$. Comme cette loi est explicite, déterminer ses régions HPD peut être fait formellement. 


\begin{proposition}{Changement de variable (rappel).}\label{changement.var}
Soit $X\sim f_X$ sur $\R$ et $Y=g(X)\sim f_Y$ avec $g$ bijectif. Alors
\begin{eqnarray*}
f_Y(y) & = & \left|g'(g^{-1}(y))\right|^{-1} f_X(g^{-1}(y)).
\end{eqnarray*}
\end{proposition}

\end{rep}
\fi

Les régions HPD sont à manier avec précaution, car elles  ne sont pas indépendantes de la paramétrisation. \\

\begin{exec}
Soit $A_{\alpha,\pi}=\left\{\theta\in\Theta  \ , \ \pi(\theta|x)\geq h_{\alpha}\right\}$ une région HPD et soit 
$$
\eta = g(\theta)
$$
un $C^1-$diffémorphisme (bijection). On définit alors la région HPD correspondante pour $\pi(\eta|x)$ :
\begin{eqnarray*}
\tilde{A}_{\alpha,\pi} & = & \left\{\eta\in g(\Theta)  \ , \ \pi(\eta|x)\geq \tilde{h}_{\alpha}\right\}
\end{eqnarray*}
\begin{itemize}
    \item Sous quelle condition peut-on écrire que $\tilde{A}_{\alpha,\pi}=g\left(A_{\alpha,\pi}\right)$ ?
    \item Illustrons cela en supposant $X\sim{\cal{N}}(\theta,1)$ et $\pi(\theta)\propto 1$, puis en posant $\eta=\exp(\theta)$.
\end{itemize}
\end{exec}

\if\mycmdexotwo1 \begin{rep}% Non invariance Régions HPD

En général, on peut constater que $\tilde{A}_{\alpha,\pi}\neq g\left(A_{\alpha,\pi}\right)$. En effet,
$$
\pi(\eta|X) = \pi(\theta(\eta)|X) \left| \frac{d\theta}{d\eta} \right|
$$
donc 
\begin{eqnarray*}
\left\{\theta \ ; \ \pi(\eta|X)\geq \tilde{h}_{\alpha}\right\} &= & \left\{\theta \ ; \ \pi(\theta(\eta)|X)\geq \tilde{h}_{\alpha}  \left| \frac{d\theta}{d\eta} \right|^{-1}\right\} 
\end{eqnarray*}
et on a donc égalité si et seulement si
\begin{eqnarray*}
\tilde{h}_{\alpha}  \left| \frac{d\theta}{d\eta} \right|^{-1} & = & {h}_{\alpha}.
\end{eqnarray*}
Si l'on suppose par exemple que $X\sim{\cal{N}}(\theta,1)$ et $\pi(\theta)\propto 1$, alors
\begin{eqnarray*}
\theta|X & \sim & {\cal{N}}(X,1).
\end{eqnarray*}
Posons alors $\eta=\exp(\theta)$. 
Comme $\pi(\theta|X)\geq h_{\alpha} \Leftrightarrow (\theta-X)^2 \leq \left(\phi^{-1}(1-\alpha)\right)^2$, alors
\begin{eqnarray*}
A_{\alpha,\pi} & = & \left[X-\phi^{-1}(1-\alpha) \ ; \ X+\phi^{-1}(1-\alpha)\right].
\end{eqnarray*}
Par ailleurs, 
\begin{eqnarray*}
\pi(\eta|X) & \propto & \frac{1}{\eta} \exp\left(-\frac{1}{2}\left(X-\log \eta \right)^2 \right).
\end{eqnarray*}
Donc
\begin{eqnarray*}
\pi(\eta|X)  \geq  \tilde{h}_{\alpha}  & \Leftrightarrow & \frac{1}{2}\left(X-\log \eta \right)^2 + \log \eta \leq {h}_{\alpha}, \\
&  \Leftrightarrow & \frac{1}{2}\left(2X-\log \eta\right)^2 - \frac{1}{2} X^2 \leq {h}_{\alpha}, \\
& \Leftrightarrow & \left(2X-\log \eta\right)^2 \leq 2{h}_{\alpha}, \\
& \Leftrightarrow &  \log\eta \in \left[2X-\phi^{-1}(\alpha/2) \ , \ 2X+\phi^{-1}(\alpha/2)\right]
\end{eqnarray*}
Ainsi, le lien entre $\tilde{A}_{\alpha,\pi}$ et $A_{\alpha,\pi}$ ne correspond pas à
la transformation initiale (exponentielle). 
\end{rep}
\fi

\vspace{1cm}

Nous pouvons comprendre pourquoi une région de confiance n’est pas invariante par reparamétrisation. En effet, cette région se définit comme une solution du problème de minimisation suivant :
$$
A_{\alpha,\pi} = \arg\min\limits_{A, \Pi(A|X)\geq 1-\alpha} \mbox{Vol}(A)
$$
où $\mbox{Vol}(A)=\int_A d\mu(\theta)$. Or la mesure de Lebesgue n’est 
pas invariante par reparamétrisation. Une idée pour lever cette difficulté est donc logiquement d’abandonner la mesure de Lebesgue et de considérer pour une mesure $s$ :
\begin{eqnarray*}
A_{\alpha,\pi,s} = \arg\min\limits_{A, \Pi(A|X)\geq 1-\alpha} \int_A d s(\theta).
\end{eqnarray*}

\subsubsection*{Calcul de régions HPD}

Pour calculer les régions HPD, il y a plusieurs méthodes :
\begin{enumerate}
\item \emph{Méthode analytique et numérique} : c’est ce qui a été fait lors de l’exemple précédent. Précisons une nouvelle fois que cette méthode ne peut s’appliquer que dans des cas assez rares.

\item \emph{Méthode par approximation} : cette méthode peut être appliquée si le modèle est régulier. L'usage du théorème de Bernstein-von Mises permet d'approximer la loi {\it a posteriori} par une gaussienne. On retombe peu ou prou sur des régions HPD proches de celles du maximum de vraisemblance.
\item \emph{Méthode par simulation}. En effet, une région $\alpha-$crédible peut génériquement \^etre estimée par les quantiles empiriques de la \textcolor{black}{simulation {\it a posteriori}} (voir plus loin).

\begin{theorem}
Supposons avoir un échantillon iid $\theta_1,\ldots,\theta_m\sim \pi(\theta|x_1,\ldots,x_n)$ avec $\theta\in\R$. Alors les intervalles de quantiles empiriques de la forme $[\theta^{(\alpha/2)},\theta^{(1-\alpha/2)}]$ sont tels que
\begin{eqnarray*}
\Pi\left(\theta\in\left[\theta^{(\alpha/2)},\theta^{(1-\alpha/2)}\right]|x_1,\ldots,x_n\right) & \xrightarrow[]{m\to\infty} 1-\alpha.
\end{eqnarray*}
\end{theorem}

Il n'est cependant pas garanti qu'une telle région soit HPD.
Pour $m$ grand, $\theta^{(\alpha/2)}$ s’approche du quantile d’ordre $\alpha/2$ de la loi {\it a posteriori}. Cette région n’est pas nécessairement HPD mais reste $\alpha-$crédible. Cette méthode est particulièrement adaptée lorsque la loi {\it a priori} est unimodale. Il est toujours utile de représenter graphiquement les sorties pour fixer les idées. Enfin, il est aussi envisageable d’avoir recours à une estimation non paramétrique par noyaux.
\end{enumerate}



%%%%%%%%%%%%%%%%%%% TP avec correction %%%%%%%%%%%%%%%
\clearpage
 \subsection{TP : Comparatif des approches bayésienne et fréquentiste pour les régions HPD }

Soit $x_1,\ldots,x_n$ des réalisations iid de loi ${\cal{N}}(\mu,\sigma^2)$. On choisit la mesure {\it a priori} (non probabiliste) jointe
\begin{eqnarray*}
\pi(\mu,\sigma^2) & \propto & 1/\sigma^2.
\end{eqnarray*}
\begin{enumerate}
\item Déterminez la loi {\it a posteriori} jointe $\pi(\mu,\sigma^2|x_1,\ldots,x_n)$ 
\item Déterminez la loi {\it a posteriori} marginale $\pi(\mu|x_1,\ldots,x_n)$ 
\item Calculez la région HPD de seuil $\alpha$ pour $\mu$ et comparez-la à la région de confiance fréquentiste, de même seuil, qu'on pourrait calculer par l'emploi du maximum de vraisemblance.
\item Déterminez la loi {\it a posteriori} marginale $\pi(\sigma^2|x_1,\ldots,x_n)$ ; le calcul de la région HPD est-il simple ?
\end{enumerate}
\paragraph{Remarque.} ``Déterminer" signifie indiquer si la loi appartient à une famille connue, par exemple largement implémentée sur machines. La connaissance des lois gamma, inverse gamma et Student est peut-être nécessaire pour répondre aux questions.

\if\mycmdtpthree1 \paragraph{\bf Réponses.} \\

\paragraph{1. Loi jointe {\it a posteriori}.}
On a 
\begin{eqnarray*}
\pi(\mu,\sigma^2|x_1,\ldots,x_n) & \propto & \sigma^{-(n+2)} \exp\left(-\sum\limits_{i=1}^n \frac{(x_i-\mu)^2}{2\sigma^2}\right), \\
& = & \sigma^{2(-(n/2+1)} \exp \left(-\sum\limits_{i=1}^n \frac{x^2_i}{2\sigma^2} +\frac{2\mu\sum\limits_{i=1}^n x_i}{2\sigma^2} - n\mu^2/2\sigma^2 \right), \\
& = & \sigma^{2(-(n/2+1)} \exp \left( - \frac{n\left(\mu-\bar{x}_n\right)^2}{2\sigma^2}
\right) \exp \left( -\frac{1}{2\sigma^2} S_n^2\right)
\end{eqnarray*}
avec
\begin{eqnarray*}
S^2_n & = & -\sum\limits_{i=1}^n \left(\bar{x}^2_n-x^2_i\right) \ = \  \sum\limits_{i=1}^n \left(\bar{x}_n-x_i\right)^2 \ \geq \ 0.
\end{eqnarray*}
\paragraph{2. Loi marginale {\it a posteriori} de $\mu$.}
On obtient alors,
%\begin{eqnarray*}
%\pi(\mu|\sigma^2,x_1,\ldots,x_n) & \propto & \exp\left(-\frac{n}{2\sigma^2}\left(\mu-\bar{x}_n\right)^2 \right) \ \equiv \ {\cal{N}}\left(\bar{x}_n, \sigma^2/n\right)
%\end{eqnarray*}
%et, 
par intégration,
\begin{eqnarray*}
\pi(\mu|x_1,\ldots,x_n) & = & \int_{\R^+} \pi(\mu,\sigma^2|x_1,\ldots,x_n) \ d\sigma^2, \\
& \propto & \int_{\R^+} \sigma^{2(-(n/2+1)} \exp \left( - \frac{n\left(\mu-\bar{x}_n\right)^2}{2\sigma^2} -\frac{1}{2\sigma^2} S_n^2
\right) \ d\sigma^2.
\end{eqnarray*}
En reconnaissant dans l'intégrande le terme général d'un loi inverse gamma ${\cal{IG}}(n/2,(S^2_n + n(\mu-\bar{x}_n)^2)/2)$, on a alors
\begin{eqnarray*}
\pi(\mu|x_1,\ldots,x_n) & \propto & \Gamma(n/2)\left(S^2_n + n(\mu-\bar{x}_n)^2\right)^{-n/2}, \\
& \propto & \left(1+k_n u^2\right)^{-(k-1)/2}
\end{eqnarray*}
avec $k=n+1$ et $u=\sqrt{k k_n}(\mu-\bar{x}_n)$ et $k_n=n/S^2_n$. %On peut alors réécrire plus simplement
%\begin{eqnarray*}
%\pi(\mu|x_1,\ldots,x_n) & \propto & \left(1+u^2/k\right)^{-(k-1)/2}
%\end{eqnarray*}
%qui définit le terme général d'une loi de Student en $u$ à $k$ degrés de liberté. 
Posons alors formellement le changement de variable 
$$
u = \frac{\sqrt{n(n+1)}}{S_n}(\mu-\bar{x}_n) \ = \ g(\mu)
$$
et procédons à un changement de variable (cf. Proposition \ref{changement.var}) pour connaître la loi de $u$. On a 
\begin{eqnarray*}
g^{-1}(u) & = & u \frac{S_n}{\sqrt{n(n+1)}} + \bar{x}_n, \\
(g'(u))^{-1} & = &  \left(\frac{\sqrt{n(n+1)}}{S_n}\right)^{-1}.
\end{eqnarray*}
Donc
\begin{eqnarray*}
\pi_U(u) & \propto & \pi_{\mu}(g^{-1}(u)), \\
& \propto & (1+u^2/k)^{-\frac{k+1}/2}.
\end{eqnarray*}
Donc la loi {\it a posteriori} de la variable $u$ est une Student à $k=n+1$ degrés de liberté :
\begin{eqnarray*}
u|x_1,\ldots,x_n & = & \frac{\sqrt{n(n+1)}}{S_n}(\mu-\bar{x}_n) \ \sim \ {\cal{S}}_t(n+1).
\end{eqnarray*}
On dit alors que la loi de $\mu$ est une \emph{Student décentrée} à $k$ degrés de liberté. 

\paragraph{3. Région HPD pour $\mu$.}
On veut alors déterminer 
\begin{eqnarray*}
{\cal{A}}_{\alpha,\pi} & = & \left\{\mu, \ \pi(\mu|x_1,\ldots,x_n ) \geq 1-\alpha\right\}.
\end{eqnarray*}
Donc en notant $t_{n+1}(\alpha)$ le quantile de seuil $\alpha$ de la loi ${\cal{S}}_t(n+1)$, on a (par symmétrie de cette loi autour de 0)
\begin{eqnarray*}
\Pi\left(-t_{n+1}(1-\alpha/2) \leq u \leq (t_{n+1}(1-\alpha/2)|x_1,\ldots,x_n\right) & = & 1-\alpha
\end{eqnarray*}
soit, de fa\c con équivalente,
\begin{eqnarray*}
\Pi\left(\bar{x}_n-\frac{S_n}{\sqrt{n(n+1)}}t_{n+1}(1-\alpha/2) \leq \mu \leq \bar{x}_n+\frac{S_n}{\sqrt{n(n+1)}}t_{n+1}(1-\alpha/2)|x_1,\ldots,x_n\right) & = & 1-\alpha.
\end{eqnarray*}
Rappelons que la région fréquentiste de seuil $1-\alpha$ connue pour l'estimation du maximum de vraisemblance (EMV), dès qu'on a pris conscience que l'EMV de $\sigma^2$ est
\begin{eqnarray*}
\hat{\sigma}^2_n & = & S^2_n/n,
\end{eqnarray*}
est :
\begin{eqnarray*}
\left[\bar{x}_n-\frac{S_n}{n}t_{n+1}(1-\alpha/2) \ ; \ \bar{x}_n+\frac{S_n}{n}t_{n+1}(1-\alpha/2)\right].
\end{eqnarray*}
Avec $n<\sqrt{n(n+1)}$ on voit que la région bayésienne est légèrement plus étroite que la région fréquentiste (il y a apport d'information avec l'ajout du prior), mais qu'il y a équivalence asymptotique (ce qui est logique). 


\paragraph{4. Loi marginale {\it a posteriori} de $\sigma^2$.} De la même fa\c con que précédemment, on a 
\begin{eqnarray*}
\pi(\sigma^2|x_1,\ldots,x_n) & \propto & \frac{1}{\sigma^{n+2}}   \exp\left(-S^2_n/2\sigma^2\right) \int_{\R} \exp\left(-\frac{n(\mu-\bar{x}_n)^2}{2\sigma^2}\right) \ d\mu, \\
 & \propto & \frac{1}{\sigma^{n+2}}   \exp\left(-S^2_n/2\sigma^2\right) \frac{\sqrt{2\pi}}{\sqrt{n}}\sigma, \\
 & \propto &  \frac{1}{\left(\sigma^2\right)^{n/2+1/2}} \exp\left(-S^2_n/2\sigma^2\right)
\end{eqnarray*}
et on reconnaît ici le terme général d'une loi inverse gamma :
\begin{eqnarray*}
\sigma^2|x_1,\ldots,x_n  & \sim & {\cal{IG}}(n/2-1/2,S^2_n/2)
\end{eqnarray*}
qui n'est définie que si $n>1$ (ce qui semble logique : il faut au moins avoir deux données pour inférer sur la variance $\sigma^2$). De plus, pour avoir $S^2_n\neq 0$ si $n=2$, il faut que $x_1\neq x_2$. Comme cette loi est explicite, déterminer ses régions HPD peut être fait formellement. 


\begin{proposition}{Changement de variable (rappel).}\label{changement.var}
Soit $X\sim f_X$ sur $\R$ et $Y=g(X)\sim f_Y$ avec $g$ bijectif. Alors
\begin{eqnarray*}
f_Y(y) & = & \left|g'(g^{-1}(y))\right|^{-1} f_X(g^{-1}(y)).
\end{eqnarray*}
\end{proposition}

\fi
