\documentclass[10pt]{article}

\usepackage[english,french]{babel}
\usepackage[latin1]{inputenc}
%\usepackage{natbib}
\usepackage{amssymb}
\usepackage{multicol}
\usepackage[fleqn]{amsmath}
\usepackage{epsfig}
\usepackage[normalem]{ulem}
\usepackage{verbatim}
\usepackage{graphicx}
\usepackage{url} % pour insérer des url
\usepackage{color}
\usepackage{bbm}
\usepackage{bm}
\usepackage{dsfont}
\usepackage{amsmath,amsfonts,times,latexsym,comment,times}
\usepackage{color,epsfig,rotating}
\newcommand{\ds}{\displaystyle}
\newcommand{\bce}{\begin{center}}
\newcommand{\ece}{\end{center}}
%\usepackage{mprocl}


\def\bx{\mathbf{x}}
\def\by{\mathbf{y}}
\def\bz{\mathbf{z}}
\def\bp{\mathbf{p}}
\newcommand{\MRTF}{\mbox{MRTF}}
\newcommand{\mttf}{\mbox{mttf}}
\newcommand{\mode}{\mbox{md}}
\newcommand{\sS}{\mbox{S}}
\newcommand{\LL}{\ell}
\newcommand{\DAC}{\mbox{DAC}}
\newcommand{\D}{\mbox{D}}
\newcommand{\R}{I\!\!R}
\newcommand{\N}{I\!\!N}
\newcommand{\Q}{\mathbbm{Q}}
\newcommand{\I}{\mathds{1}}
\newcommand{\C}{C}
\newcommand{\Pp}{\mathbbm{P}}
\newcommand{\E}{\mbox{E}}
\newcommand{\V}{\mbox{Var}}
\newcommand{\Var}{\mbox{Var}}
\newcommand{\Cov}{\mbox{Cov}}
\newcommand{\1}{\mathbbm{1}}
\newcommand{\Med}{\mbox{Med}}
\newcommand{\Mod}{\mbox{Mod}}
\newcommand{\Md}{\mbox{M}_d}
\newcommand{\Card}{\mbox{Card}}
\newcommand{\DIP}{\mbox{Dip}}
\newcommand{\Supp}{\mbox{Supp}}


\newcounter{cptpropo}[part]
\newenvironment{propo}[0]
{\noindent\textsc{Proposition}\,\refstepcounter{cptpropo}\thecptpropo.\it}

\newcounter{cptlemmo}[part]
\newenvironment{lemmo}[0]
{\noindent\textsc{Lemma}\,\refstepcounter{cptlemmo}\thecptlemmo.\it}

\newcounter{cptexo}[part]
\newenvironment{exo}[0]
{\noindent\textsc{Example}\,\refstepcounter{cptexo}\thecptexo.\it}

\newtheorem{theorem}{Theorem}
\newtheorem{definition}{Definition}
\newtheorem{proposition}{Proposition}
%\newtheorem{proof}{Proof}
%\renewcommand{\theproof}{\empty{}} 
\newtheorem{lemma}[theorem]{Lemma}
\newtheorem{corollary}{Corollary}
\newtheorem{assumption}{\noindent Assumption}
\newtheorem{acknowledgments}{\noindent Acknowledgments}
\newtheorem{example}{\noindent Example}
\newtheorem{remark}{\noindent Remark}


\title{Examen 2020 : Modélisation et statistique bayésienne computationnelle \\ -- Corrig\'e -- }
\date{30 mars 2020}



\begin{document}

%%%%%%%%%%%%%%%%%%
\maketitle

%%%%%%%%%%%%%%%%%%  
 




{\it L'examen dure en théorie 3h et est noté sur 30. Tous les supports de cours sont autorisés. Il est attendu un code R ou Python commenté {\it a minima} pour les parties computationnelles, et un support papier peut être utilisé pour la partie formelle. Lisez bien tout le document, certaines questions peuvent être traitées formellement indépendamment du reste de l'exercice qui les contient.} \\

{\it Vous pouvez rendre ce travail par email à l'adresse} 
\begin{center}
 \texttt{nicolas.bousquet@sorbonne-universite.fr} 
 \end{center} 
{\it ou }
 \begin{center}
 \texttt{nbousquet@gmail.com} 
  \end{center} 
{\it avant le {\bf 31 mars à 12h00 (midi)}. Si vous ne pouvez scanner vos écrits, prenez des photos, constituez un répertoire zippé avec vos photos triées par ordre} \\

 {\it Vous pouvez rédiger en fran\c cais ou en anglais.} \\ 


\section{Construction de prior}

Soientt $X$ une variable aléatoire de loi de Poisson ${\cal{P}}(\theta)$ avec $\theta\in\R^+_*$, et $x_1,\ldots,x_n$ un échantillon de cette loi.
\begin{enumerate}
\item Déterminer la mesure {\it a priori} de Jeffreys $\pi^J(\theta)$.
\item \'Evaluer, à partir de l'existence des lois {\it a posteriori} si cette mesure {\it a priori} est préférable à la mesure {\it a priori} invariante par transformation d'échelle $\pi_0(\theta)\propto1/\theta$.
\item Soit $\pi_{\alpha}(\theta) \propto \theta^{-\alpha}$ avec $\alpha\in\R^+$. Donner l'expression de la fonction de masse prédictive {\it a posteriori} $P_{\alpha}(X=k|x_1,\ldots,x_n)$ ainsi que son espérance et sa variance, et leurs conditions d'existence.
%\item Donner la loi {\it a priori} d'entropie maximale définie pour la mesure de référence $\pi_{\alpha}(\theta)$ et les contraintes $\E_{\pi_0}[\theta]=1$, $\V_{\pi_0}[\theta]=1$. Discuter l'existence d'une vraie densité {\it a priori} en fonction de $\alpha$
\end{enumerate}

\paragraph{\bf Réponses.}
\begin{enumerate}
\item La fonction de masse (densité) de $X\sim {\cal{P}}(\theta)$ est, $\forall k\in\N$, 
$$
P(X=k|\theta) = \frac{\theta^k}{k!}\exp(-\theta) 
$$
et donc
\begin{eqnarray*}
\frac{\partial^2 \log P(X=k|\theta)}{\partial \theta^2} &= & -k/\theta^2
\end{eqnarray*}
Par absolue continuité l'information de Fisher est donc $I(\theta)=-\E_X\left[\frac{\partial^2 \log P(X=k|\theta)}{\partial \theta^2}\right]$ et donc, avec $\E_X[k]=\theta$, il vient  $I(\theta)=1/\theta$ et par définition
\begin{eqnarray*}
\pi^J(\theta) & \propto & \theta^{-1/2}.
\end{eqnarray*}
\item On a alors
\begin{eqnarray*}
\pi^J(\theta|x_1,\ldots,x_n) & \propto & \theta^{\sum_i x_i} \exp\left(-n\theta\right) \theta^{-1/2}
\end{eqnarray*}
on reconnaît le terme général d'une loi gamma ${\cal{G}}\left(\sum_i x_i+1/2,n\right)$ qui est propre (intégrable et donc bien définie) $\forall x\in\N$ et tout $n\geq 1$. Si on remplace $\pi^J(\theta)$ par $\pi_0(\theta)$, alors la loi {\it a posteriori} devient ${\cal{G}}\left(\sum_i x_i,n\right)$ qui peut ne pas être définie si tous les $x_i$ valent 0 (\c ca serait le cas, typiquement, si $\theta\ll 1$). La mesure de Jeffreys est donc préférable.
\item On a, par définition
\begin{eqnarray}
P_{\alpha}(X=k|x_1,\ldots,x_n) & = & \int_{\R^+_*} P(X=k|\theta)\pi_{\alpha}(\theta|x_1,\ldots,x_n)\ d\theta \label{eq1}
\end{eqnarray}
et si 
\begin{eqnarray}
\alpha & < & \sum_i x_i+1, \label{cond1}
\end{eqnarray}
alors
\begin{eqnarray}
\pi_{\alpha}(\theta|x_1,\ldots,x_n) & \equiv & {\cal{G}}\left(\sum_i x_i-\alpha+1,n\right) \label{eq2}
\end{eqnarray}
et 
\begin{eqnarray*}
\E[X|x_1,\ldots,x_n] &= & \E_{\pi_{\alpha}}\left[\E[X|\theta] | x_1,\ldots,x_n\right], \\
& = &  \E_{\pi_{\alpha}}\left[\theta | x_1,\ldots,x_n\right], \\
& = & \frac{\sum_i x_i-\alpha+1}{n}.
\end{eqnarray*}
De plus 
\begin{eqnarray*}
\E[X^2|x_1,\ldots,x_n] &= & \E_{\pi_{\alpha}}\left[\E[X^2|\theta] | x_1,\ldots,x_n\right], \\
& = &  \E_{\pi_{\alpha}}\left[\theta + \theta^2| x_1,\ldots,x_n\right], \\
& = & \frac{\sum_i x_i-\alpha+1}{n} + \V_{\pi_{\alpha}}\left[\theta| x_1,\ldots,x_n\right] + \left(\E_{\pi_{\alpha}}\left[\theta| x_1,\ldots,x_n\right]\right)^2, \\
& = & \frac{\sum_i x_i-\alpha+1}{n} + \frac{\sum_i x_i-\alpha+1}{n^2} + \left(\frac{\sum_i x_i-\alpha+1}{n}\right)^2
\end{eqnarray*}
Donc
\begin{eqnarray*}
\V[X|x_1,\ldots,x_n] &= & \frac{\sum_i x_i-\alpha+1}{n} + \frac{\sum_i x_i-\alpha+1}{n^2}, \\
& = & \frac{\sum_i x_i-\alpha+1}{n}\left(1+1/n\right).
\end{eqnarray*}
Enfin, en reprenant (\ref{eq1}) et (\ref{eq2}), on obtient 
\begin{eqnarray*}
P_{\alpha}(X=k|x_1,\ldots,x_n) & = & \int_{\R^+_*} \frac{\theta^k}{k!}\exp(-\theta) \frac{n^{\sum_i x_i-\alpha+1}}{\Gamma(\sum_i x_i-\alpha+1)} \theta^{\sum_i x_i-\alpha} \exp\left(-n\theta \right) \ d\theta, \\
& = &  \frac{n^{\sum_i x_i-\alpha+1}}{k!\Gamma(\sum_i x_i-\alpha+1)}  \int_{\R^+_*} \theta^{k + \sum_i x_i-\alpha} \exp\left(-(n+1)\theta \right) \ d\theta
\end{eqnarray*}
et on reconnaît sous l'intégrale le terme général d'une loi ${\cal{G}}\left(k+\sum_i x_i-\alpha+1,n+1\right)$, bien définie sous la condition (\ref{cond1}). L'intégration donne la constante de normalisation de cette densité et donc
\begin{eqnarray*}
P_{\alpha}(X=k|x_1,\ldots,x_n) & = & \frac{n^{\sum_i x_i-\alpha+1}}{k!\Gamma(\sum_i x_i-\alpha+1)} \frac{\Gamma(k+\sum_i x_i-\alpha+1)} {(n+1)^{k+\sum_i x_i-\alpha+1}}
\end{eqnarray*}
qui peut éventuellement se simplifier en utilisant (par exemple) la relation
$$
\Gamma(x+k) = \frac{(x-1)!}{(x+k-1)!} \Gamma(x).
$$
ou une formule de type Stirling. 
%\item La loi d'entropie maximale par rapport à $\pi^J(\theta)$ s'écrit
%\begin{eqnarray*}
%\pi(\theta) & \propto & \pi^J(\theta)\exp(\lambda_1 \theta + \lambda_2 \theta^2\right)
%\end{eqnarray*}
%car la contrainte en variance se réécrit comme la contrainte linéaire 
%\begin{eqnarray*}
%\E[\theta^2] & = & \V[\theta] + \E^2[\theta] \ = \ 2.
%\end{eqnarray*}
\end{enumerate}



\section{Risque d'un estimateur}

Considérons une variable binomiale $X\sim{\cal{B}}(n,p)$ de probabilité $p\in[0,1]$. Soit la perte quadratique $L(\delta,p)$. On appelle {\it risque bayésien d'un estimateur $\delta(x)$} la quantité $\E_{\pi}[L(\delta(x),p)|x]$, et  {\it risque fréquentiste de $\delta(x)$}  la quantité $\E_{X}[L(\delta(x),p)]$.
\begin{enumerate}
\item Soit $\pi(p)$ le prior de Laplace. Définissez l'estimateur MAP ({\it maximum a posteriori}) $\delta_1(x)$ de $p$.
\item En choisissant plutôt $\pi(p)$ comme le prior de Jeffreys, calculez les risques bayésien et fréquentiste $R_b(x)$ et $R_f(p)$ de  $\delta_1(x)$.
\item Comparez $r_f = \sup_p R_f(p)$ à 
$r_b  =  \sup_x R_b(x)$. 
\end{enumerate}

\paragraph{\bf Réponses.}
\begin{enumerate}
\item L'estimateur MAP pour le prior de Laplace (loi uniforme) est le mode de la distribution {\it a posteriori}
$$
p|x \sim {\cal{B}}_e(1+x,n-x+1)
$$
c'est-à-dire
$$
\delta_1(x) = x/n.
$$
\item On a 
\begin{eqnarray*}
R_b(x) &= & \E_{\pi}\left[(\delta_1(x)-p)^2|x\right] \ = \ \E_{\pi}\left[(x/n-p)^2|x\right], \\
& = & \left(\frac{x+1/2}{n+1} - \frac{x}{n}\right)^2 + \frac{(x+1/2)(n-x+1/2)}{(n+1)^2(n+2)}, \\
& = & \frac{(x-n/2)^2}{(n+1)^2 n^2} + \frac{(x+1/2)(n-x+1/2)}{(n+1)^2(n+2)}
\end{eqnarray*}
car $\pi(p)$ est la loi ${\cal{B}}_e(1/2,1/2)$ (loi de Jeffreys) et donc
$$
p|x \sim {\cal{B}}_e(1/2+x,n-x+1/2). 
$$
De plus, 
\begin{eqnarray*}
R_f(p) & = & \E_X\left[\left(\delta_1(x)- p\right)^2\right], \\
& = & \V\left[x/n\right], \\
& = & \frac{p(1-p)}{n}.
\end{eqnarray*}
\item Il est aisé de voir que $r_f =(4n)^{-1}$ et 
\begin{eqnarray*}
r_b  & = & \left\{4(n+2)\right\}^{-1}.
\end{eqnarray*}
Ainsi, $r_b< r_f$. 
\end{enumerate}


\section{Maximisation d'entropie }\label{max.entropie}

On définit une nouvelle méthodologie de construction de prior de la fa\c con suivante. \'Etant donné un modèle d'échantillonnage $X|\theta \sim p(x|\theta)$, avec $x\in S$ et $\theta\in\Theta\in\R^d$, et un prior de référence $\pi^J(\theta)$, on définit
\begin{eqnarray}
\pi^*(\theta) & = & \arg\max\limits_{\pi(\theta)\geq 0} G(\Theta) \label{mdiprior}
\end{eqnarray}
où $G(\Theta)$ est l'information moyenne apportée par la densité $p$ relativement à celle apportée par un prior $\pi(\theta)$ :
\begin{eqnarray*}
G(\Theta) & = & \E_{\theta}\left[H^J(\Theta) - H(X|\theta)\right],
\end{eqnarray*}
où $H(X|\theta)$ et
$H^J(\Theta)$ sont respectivement l'entropie (relative à une mesure de Lebesgue) du modèle d'échantillonnage et l'entropie (relative à $\pi^J(\theta)$) du prior $\pi(\theta)$. 
\begin{enumerate}
\item Prouvez que si $Y\sim f(y)$ sur un espace normé et mesuré $\Omega\in\R^q$ avec $q<\infty$ et $f\in L^2(\Omega)$, alors l'entropie relative à la mesure de Lebesgue de $f$ est bornée. 
\item Prouvez que le problème  (\ref{mdiprior}), en imposant la contrainte que $p(x|\theta)$ et $\pi(\theta)$ soient respectivement $L^4$ ($\forall \theta\in\Theta$) sur $S$ et sur $\Theta$,  implique que $\pi(\theta)$ est solution du problème de maximum d'entropie de $\pi(\theta)$ sous une contrainte linéaire
\begin{eqnarray}
\int_{\Theta} Z(\theta) \pi(\theta) \ d\theta & = & c \ < \ \infty \label{contr2}
\end{eqnarray}
où $Z(\theta)$ est l'information de Shannon (ou entropie différentielle négative) de $p$ 
\begin{eqnarray*}
Z(\theta) & = & \int_{S} p(x|\theta) \log p(x|\theta) \ dx
\end{eqnarray*}
et $c$ prend une valeur maximale (mais finie). 
\item Pour $S=\R^+$ et $(\beta,\eta)\in\R^+_* \times \R^+_*$, considérons maintenant la loi de fonction de répartition de Weibull
\begin{eqnarray*}
P(X<x|\theta) & = & 1-\exp\left(-\left\{\frac{x}{\eta}\right\}^{\beta}\right).
\end{eqnarray*}
\begin{enumerate}
\item Calculez $Z(\eta,\beta)$ pour ce modèle.
\item En utilisant le prior de Berger-Bernardo $\pi^J(\eta,\beta)\propto (\eta,\beta)^{-1}$ comme mesure de référence, donnez la solution formelle $\pi^*(\eta,\beta)$  du problème de maximisation d'entropie relative sous les contraintes (\ref{contr2}) et
\begin{eqnarray}
\int_S x m_{\pi}(x) \ dx & = & x_e \label{cons2}
\end{eqnarray}
où $m_{\pi}(x)$ est la loi {\it a priori} prédictive. 
\item Placez les résultats sous la forme hiérarchique 
\begin{eqnarray*}
\pi^*(\theta) & = & \pi^*(\eta|\beta)\pi^*(\beta).
\end{eqnarray*}
et prouvez que la loi {\it a priori} sur $\beta$ peut s'écrire
\begin{eqnarray*}
\pi^*(\beta) & \propto & \tilde{\pi}^*(\beta) 
\end{eqnarray*}
avec
\begin{eqnarray}
\tilde{\pi}^*(\beta) & = & \frac{\beta^{-\lambda_1-1}\exp\left(-\lambda_1 \frac{\gamma}{\beta}\right)}{\Gamma^{\lambda_1}(1+1/\beta)}\label{pistar}
\end{eqnarray}
où $\lambda_1$ est un multiplicateur de Lagrange. 
\item En pla\c cant des contraintes sur les multiplicateurs de Lagrange issus de l'écriture générale de $\pi^*(\eta,\beta)$, reconnaissez-vous une forme spécifique (connue) pour $\pi^*(\eta|\beta)$ et $\pi^*(\beta)$ ? La loi $\pi^*(\eta|\beta)$ est-elle conjuguée conditionnellement à $\beta$ ? 
\item Cette loi jointe $\pi^*(\theta)$ est-elle propre (intégrale) ? Sous quelle(s) condition(s) sur les multiplicateurs de Lagrange ?
\item Reliez formellement les multiplicateurs de Lagrange à $x_e$ en vérifiant l'équation (\ref{cons2}). Doit-on connaître la constante d'intégration de $\pi^*(\beta)$ pour ce faire ? 
\item Proposez, codez et validez une méthode numérique permettant de simuler des tirages de $\beta$ selon $\pi^*(\theta)$ (formule (\ref{pistar})), en fixant $\lambda_1=1$. Pour la validation, utilisez plutôt la représentation de la variable $Y=1/beta$ en opérant un changement de variable.
\end{enumerate}
\end{enumerate}

\textit{
\paragraph{\bf Indications.}
\begin{itemize}
\item Il peut être utile de prouver au préalable  à la question 1 que $\log y \leq 1 + y$ $\forall y\in \R^+_*$
\item On rappelle que l'espérance de la loi de Weibull est 
\begin{eqnarray}
\E[X|\theta] & = & \eta\Gamma(1+1/\beta) \label{weibu}
\end{eqnarray}
et que lorsque $\beta>0$ 
\begin{eqnarray}
\Gamma(1+1/\beta) & \geq & \frac{\sqrt{\pi}}{3}  \label{borne.min}
\end{eqnarray}
\item On rappelle les formules suivantes :
\begin{eqnarray}
\int_0^{\infty} (\log x) \exp(-x) \ dx & = & -\gamma \label{aide1} \\
\int_0^{\infty} x \exp(-x) \ dx & = & \Gamma(2) \label{aide2}
\end{eqnarray}
où $\gamma$ est la constante d'Euler (que vous pouvez prendre égale à 0.5772157)
\end{itemize}
}


\paragraph{\bf Réponses.}
\begin{enumerate}
\item L'entropie relative à la mesure de Lebesgue sur $\Omega$ est $H(f)=-\E_f[\log f] $ et, via l'inégalité de Jensen ($-\log $ étant convexe), 
\begin{eqnarray*}
-\E_f[\log f] & \leq & -\log \E_f[ f] \ = \ -\int_{\Omega} f^2(y) \ dy
\end{eqnarray*}
et $\int_{\Omega} f^2(y) \ dy < \infty$ puisque  $f\in L^2(\Omega)$. De plus, il est aisé de montrer que  $\|og y \leq 1 + y$ $\forall y\in \R^+_*$. On a donc que
\begin{eqnarray*}
-\E_f[\log f]  & \geq & -1 - \int_{\Omega} f^2(y) \ dy.
\end{eqnarray*}
Donc $H(f)$ est bornée. 
\item La définition (\ref{mdiprior}) est celle proposée par Arnold Zellner pour définir la classe des {\it Maximal Data Information} (MDI) {\it Priors}, qui constitue une alternative souvent intéressante aux {\it reference priors} de Berger-Bernardo. On voit facilement que (modulo l'existence des intégrales ci-dessous) 
 \begin{eqnarray*}
\pi^*(\theta) & = & \arg\max\limits_{\pi(\theta)\geq 0}  \int_{\Theta} Z(\theta) \pi(\theta) \ d\theta - \int_{\Theta}
\pi(\theta) \log \frac{\pi(\theta)}{\pi^J(\theta)}. 
\end{eqnarray*}
Les deux problèmes de maximisation ont le même lagrangien si l'intégrale $\int_{\Theta} Z(\theta) \pi(\theta) \ d\theta $ est finie. Or on a
\begin{eqnarray*}
\E_{\pi}[Z] & = & \int_{\Theta} Z(\theta) \pi(\theta) \ d \theta \ = \ H(\Theta) - H(X,\Theta)
\end{eqnarray*}
où $H(\Theta)$ est l'entropie (non relative) de $\pi(\theta)$ et  $H(X,\Theta)$ est l'entropie (non relative) de la loi jointe de $(X,\theta)$. Si $\pi(\theta)$ est $L^4$ sur $\Theta$, alors elle est aussi $L^2$ sur $\Theta$ et  $H(\Theta)$ est bornée d'après la question 1. Il suffit alors de montrer que $- H(X,\Theta)$ est fini. On a (en utilisant les résultats précédents)
\begin{eqnarray*}
- H(X,\Theta) & = & \int_{S \times \Omega} p(x|\theta) \pi(\theta) \log \left\{ p(x|\theta) \pi(\theta) \right\} \ dx d\theta \\
& \leq & 1 + \int_{S \times \Omega} \left(p(x|\theta) \pi(\theta)\right)^2  \ dx d\theta \ \ \ \text{par Jensen}, \\
& \leq & 1 + \sqrt{\int_S p^4(x|\theta) \ dx}\sqrt{\int_{\Theta} \pi^4(\theta) \ d\theta}
\end{eqnarray*}
d'après l'inégalité de  Cauchy-Schwarz. Le terme de droite étant alors fini d'après les hypothèses, on a donc que $-H(X,\Theta)$ est fini et dont $\E_{\pi}[Z] $ est fini. Donc $\exists c<\infty$ tel que 
\begin{eqnarray*}
\int_{\Theta} Z(\theta) \pi(\theta) \ d\theta & = & c.
\end{eqnarray*}
\item Rappelons que la densité correspondante de Weibull s'écrit
\begin{eqnarray*}
f(x|\theta) & = & \frac{\beta}{\eta} \left(\frac{x}{\eta}\right)^{\beta-1} \exp\left(-\left\{\frac{x}{\eta}\right\}^{\beta}\right).
\end{eqnarray*}
\begin{enumerate}
\item On a, après un simple développement,
\begin{eqnarray*}
Z(\eta,\beta) & = & \log \frac{\beta}{\eta^{\beta}} + (\beta-1) \E_X\left[\log X\right] -  \E_X\left[\left(\frac{X}{\eta}\right)^{\beta}\right]. 
\end{eqnarray*}
En utilisant la transformation $u=(x/\eta)^{\beta}$, avec $du/dx = \beta x^{\beta-1}/\eta^{\beta}$, il vient
\begin{eqnarray*}
\E_X\left[\log X\right] & = & \log(\eta) \int_{0}^{\infty} \exp(-u) \ du + \frac{1}{\beta} \int_{0}^{\infty} (\log u)  \exp(-u) \ du, \\
& = &  \log(\eta) - \gamma/\beta \ \ \ \ \text{en utilisant (\ref{aide1})},
\end{eqnarray*}
puis
\begin{eqnarray*}
\E_X\left[\left(\frac{X}{\eta}\right)^{\beta}\right] & = &  \int_{0}^{\infty} u \exp(-u) \ du  \\
& = &  \Gamma(2) \ = \ 1 \ \ \ \ \text{en utilisant (\ref{aide2})}.
\end{eqnarray*}
On en déduit que
\begin{eqnarray*}
Z(\eta,\beta) & = & \log \beta - \log \eta + \gamma/\beta - (1+\gamma). 
\end{eqnarray*}
\item En notant que $\theta=(\eta,\beta)$, rappelons que l'on peut écrire la contrainte sous forme linéaire (par rapport à $\pi(\theta)$)
\begin{eqnarray}
\int_S x m_{\pi}(x) \ dx & = & x_e \ = \ \int_{\Theta} \E[X|\theta] \pi(\theta)  \ d\theta \label{mean1}
\end{eqnarray}
avec $\E_{\theta}[X]=\eta\Gamma(1+1/\beta)$ d'après (\ref{weibu}). 
\end{enumerate}
Alors la solution du problème de maximisation d'entropie s'écrit, en introduisant $(\lambda_1,\lambda_2)\in\R^2$ des multiplicateurs de Lagrange,
\begin{eqnarray*}
\pi^*(\theta) & \propto & \pi^J(\theta) \exp\left(-\lambda_1 Z(\theta) - \lambda_2\E[X|\theta]\right), \\
& \propto &  \beta^{-\lambda_1-1} \eta^{\lambda_1-1} \exp\left(-\lambda_2 \eta\Gamma(1+1/\beta)\right) \exp\left(-\lambda_1 \frac{\gamma}{\beta}\right)
\end{eqnarray*}
\item Si on impose $(\lambda_1,\lambda_2)\in\R^+\times \R^+$, on peut alors écrire
\begin{eqnarray*}
\pi^*(\theta) & = & \pi^*(\eta|\beta)\pi^*(\beta)
\end{eqnarray*}
avec
\begin{eqnarray*}
\eta|\beta & \sim & {\cal{G}}\left(\lambda_1,\lambda_2 \Gamma(1+1/\beta)\right), \\
\pi^*(\beta) & \propto & \underbrace{\frac{\beta^{-\lambda_1-1}\exp\left(-\lambda_1 \frac{\gamma}{\beta}\right)}{\Gamma^{\lambda_1}(1+1/\beta)}}_{\tilde{\pi}^*(\beta)}
\end{eqnarray*}
où ${\cal{G}}$ désigne une loi gamma \\
{\it (le terme en dénominateur de $\pi^*(\beta)$ correspondant à la constante d'intégration (à un coefficient près) de $\pi^*(\eta|\beta)$)}

\item On remarque que la loi $\pi^*(\eta|\beta)$ n'est pas conjuguée conditionnellement à $\beta$ (il faudrait que ce soit une inverse gamma, et non une gamma). 
\item Pour que la loi jointe soit propre, sachant $(\lambda_0,\lambda_1)\in\R^+_*\times \R^+_*$, il suffit donc de montrer que $\pi^*(\beta)$ est intégrable. Or, pour tout $\beta>0$, d'après (\ref{borne.min}) on a $\Gamma(1+1/\beta)\geq \sqrt{\pi}/{3}$. Donc,  
\begin{eqnarray*}
0 & \leq \ \tilde{\pi}^*(\beta) \ \leq & \left(\frac{3}{\sqrt{\pi}}\right)^{\lambda_1} \beta^{-\lambda_1-1} \exp\left(-\lambda_1 \frac{\gamma}{\beta}\right)
\end{eqnarray*}
qui est clairement intégrable. Le prior est donc propre sous les conditions $(\lambda_1,\lambda_2)\in\R^+\times \R^+$. 
\item On note $A(\lambda_1)$ la constante d'intégration de $\pi^*(\beta)$, telle que
$$
\pi^*(\beta) = A^{-1}(\lambda_1)\tilde{\pi}^*(\beta)
$$
Vérifions l'équation (\ref{cons2}) en utilisant l'expression (\ref{mean1}) : 
\begin{eqnarray*}
x_e \ = \ \int_{\Theta} \E[X|\theta] \pi(\theta)  \ d\theta & = &  A^{-1}(\lambda_1) \int_{\R^+} \Gamma(1+1/\beta) \tilde{\pi}^*(\beta)\E[\eta|\beta] \ d\beta, \\
& = & A^{-1}(\lambda_1) \int_{\R^+}  \tilde{\pi}^*(\beta) \frac{\lambda_1}{\lambda_2 \Gamma(1+1/\beta)} \ d\beta, \\
& = & \frac{\lambda_1}{\lambda_2}.
\end{eqnarray*}
Ce résultat est indépendant de la constante d'intégration de $\pi^*(\beta)$.
\item Dans ce cas unidimensionnel, l'idée la plus simple consiste à utiliser une {\bf méthode d'acceptation-rejet}, qui permettra en outre de calculer numériquement la constante d'intégration  $A(\lambda_1)$. Pour ce faire, la forme du terme général $\tilde{\pi}^*(\beta)$, proche d'une inverse gamma, peut nous inspirer. Si on choisit comme loi instrumentale la densité
\begin{eqnarray*}
g(\beta) & \equiv & {\cal{IG}}(\lambda_1,\lambda_1/\gamma),
\end{eqnarray*}
alors il vient
\begin{eqnarray*}
\frac{\tilde{\pi}^*(\beta)}{g(\beta)} & = & \frac{\Gamma^{\lambda_1/\gamma}(\lambda_1)}{\Gamma^{\lambda_1}(1+1/\beta)}\frac{1}{\left(\lambda_1/\gamma\right)^{\lambda_1}}, \\
& \leq &  \left(\frac{3}{\sqrt{\pi}}\right)^{\lambda_1} \frac{\Gamma^{\lambda_1/\gamma}(\lambda_1)}{\left(\lambda_1/\gamma\right)^{\lambda_1}}, \\
\end{eqnarray*}
d'après (\ref{borne.min}), borne supérieure qui ne dépend plus de $\beta$. En utilisant la valeur $\lambda_1=\gamma$, on obtient
\begin{eqnarray*}
\frac{\tilde{\pi}^*(\beta)}{g(\beta)} & \leq & K  \ = \ \left(\frac{3}{\sqrt{\pi}}\right)^{\gamma} \Gamma(\gamma) \ \simeq \ 2.092
\end{eqnarray*}
Le programme attendu (exemple en R fourni sur le fichier \texttt{AR.r}) doit donc mettre en \oe{}uvre le pseudo-code suivant :
\texttt{
\begin{enumerate}
\item Simuler $\beta\sim g(\beta)$.
\item Simuler $U\sim {\cal{U}}[0,1]$.
\item Accepter $\beta$ si $U\leq \frac{\tilde{\pi}^*(\beta)}{Kg(\beta)}$
\end{enumerate}
}
et en notant $p$ la proportion d'acceptation dans cet algorithme, on peut estimer $A(\lambda_1)$ par
\begin{eqnarray*}
\hat{A}(\lambda_1) & = & 1/(Kp)
\end{eqnarray*}
puis représenter la densité ${\pi}^*(\beta)$ estimée par $\hat{A}^{-1}(\lambda_1)\tilde{\pi}^*(\beta)$ et la comparer avec l'histogramme des simulations acceptées. Il est plus simple visuellement de représenter plutôt la densité de la variable $Y=1/\beta$, telle que $du = -u^2 d\beta$. La formule de changement de variable donne :
\begin{eqnarray*}
\pi^*_Y(Y) & = & \pi^*(\beta^{-1}(Y)))/Y^2
\end{eqnarray*}
avec $\beta^{-1}(Y)=1/Y$. 
\end{enumerate}


%%%%%%%%%%%%%%%%%%%%%%%%%%%%%%%%%%%%%%%%%%%%%%%%%%%%%%%%%%
\section{Calcul bayésien}

On reprend la loi de Weibull ${\cal{W}}(\eta,\beta)$ de l'exercice (\ref{max.entropie}) et on impose le prior suivant
\begin{eqnarray*}
\eta & \sim & {\cal{G}}(m,m/\eta_0) \\
\beta & \sim & \tilde{\pi}^*(\beta)  = \ \frac{\beta^{-\lambda_1-1}\exp\left(-\lambda_1 \frac{\gamma}{\beta}\right)}{\Gamma^{\lambda_1}(1+1/\beta)}
\end{eqnarray*}
qui est le prior (\ref{pistar}) sur $\beta$. \\

Soit l'échantillon
\begin{eqnarray*}
{\bf x_n} & = & \left\{ 103, 157 , 39 ,145 , 24  ,22 ,122, 126 , 66 , 97\right\},
\end{eqnarray*}
\begin{enumerate}
\item Proposez et implémentez une méthode permettant de générer des tirages {\it a posteriori} de $(\eta,\beta)$. 
\item Estimez numériquement l'espérance de la loi {\it a posteriori prédictive} 
\begin{eqnarray*}
p(x|{\bf x_n}) & = & \iint_{\R^+\times\R^+} p(x|\eta,\beta) \pi(\eta,\beta|{\bf x_n}) \ d\eta d\beta,,
\end{eqnarray*}
 en prenant $m=2$, $\eta_0=100$ et $\lambda_1=1$ 
\end{enumerate}

\paragraph{\bf Réponse.} 
\begin{enumerate}
\item Le prior n'étant pas conjugué pour aucune des deux dimensions, il est naturel de proposer un algorithme de Gibbs dont les deux simulations (de $\eta$ puis $\beta$) sont menées par des échantillonnages de Metropolis-Hastings. Cela d'autant plus qu'on n'a pas besoin de connaître la constante d'intégration de $\pi(\beta)$ (et donc d'avoir résolu complètement l'exercice (\ref{max.entropie}).  Le code-solution produit sur le fichier \texttt{calcul-bayesien.r} utilise deux marches aléatoires comme lois instrumentales. 
\item On obtient numériquement, en simulant par Monte Carlo un tirage $(\eta_i,\beta_i)\sim \pi(\eta,\beta|{\bf x_n})$ puis en calculant \begin{eqnarray*}
\E\left[X|{\bf x_n}\right] & = & \E_{\pi}\left[\eta\gamma(1+1/\beta)|{\bf x_n}\right] \\
& \simeq & \frac{1}{M} \eta_i\gamma(1+1/\beta_i), \\
&  \simeq & 357
\end{eqnarray*}
\end{enumerate}

\section{Bonus}

Soit $f(x|\theta) = h(x)\exp\left(\theta\cdot x - \psi(\theta)\right)$ une distribution d'une famille exponentielle, avec $\theta\in\Theta$. Pour toute loi {\it a priori} $\pi$, prouvez que la moyenne {\it a posteriori} de $\theta$ est donnée par
\begin{eqnarray*}
\E[\theta|x] & = & \nabla \log m_{\pi}(x) - \nabla \log h(x)
\end{eqnarray*}
où $\nabla$ est l'opérateur gradient et $m_{\pi}$ est la loi marginale {\it a priori} associée à $x$.

\paragraph{\bf Réponse.} L'espérance {\it a posteriori} de $\theta$ vaut
\begin{eqnarray*}
\E[\theta|x] & = & \frac{\int_{\Theta} \theta  h(x)\exp\left(\theta\cdot x - \psi(\theta)\right) \pi(\theta) \ d\theta}{m_{\pi}(x)}, \\
& = & \left(\frac{\partial }{\partial x} \int_{\Theta} h(x)\exp\left(\theta\cdot x - \psi(\theta)\right) \pi(\theta) \ d\theta\right) \frac{1}{m_{\pi}(x)} - \left(\frac{\partial }{\partial x}  h(x)\right) \frac{1}{h(x)}, \\
&= & \frac{\partial }{\partial x} \left( \log m_{\pi}(x) - \log h(x)\right).
\end{eqnarray*}

\end{document} 